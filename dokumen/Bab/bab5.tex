\chapter{Implementasi}
\label{chap:implementasi}

Setelah rancangan perangkat lunak berhasil dibuat, tahap selanjutnya yang akan dikerjakan adalah tahap implementasi perangkat lunak. Dalam tahap implementasi perangkat lunak, rancangan perangkat lunak yang sudah dibuat akan diubah menjadi kode program yang sesuai dengan rancangan tersebut.

Pada skripsi ini, perangkat lunak identifikasi rombongan akan diimplementasikan menggunakan bahasa pemrograman C++17. C++ dipilih karena program yang dihasilkan memiliki performa yang baik dan memiliki seperangkat perkakas pemrograman yang sudah teruji sejak lama. Paradigma pemrograman yang digunakan adalah paradigma pemrograman fungsional.

\section{Modul I/O}
\label{sec:impl-io}

Modul I/O akan diimplementasikan dalam berkas \texttt{io.cpp} dan memiliki berkas \textit{header} bernama \texttt{io.h}. Berikut merupakan detil implementasi dari modul I/O berdasarkan 2 fungsionalitas modul yang sudah dibahas pada bab sebelumnya:

\begin{enumerate}
    \item \textbf{Input}
    
    \item \textbf{Output}
\end{enumerate}

\section{Modul Penerjemah}
\label{sec:impl-parser}

Modul Penerjemah akan diimplementasikan dalam berkas \texttt{parser.cpp} dan memiliki berkas \textit{header} bernama \texttt{parser.h}. 

\section{Modul Rombongan}
\label{sec:impl-rombongan}

Modul Rombongan akan diimplementasikan dalam 2 berkas yaitu \texttt{rombongan.cpp} beserta berkas \textit{header} bernama \texttt{rombongan.h} dan berkas \texttt{similarity.cpp} beserta berkas \textit{header} bernama \texttt{similarity.h}. Pemisahan tersebut dilakukan untuk menegaskan pemisahan antara implementasi algoritma yang sudah ada dan implementasi algoritma yang diusulkan.

\section{Struktur Data}
\label{sec:impl-struct}