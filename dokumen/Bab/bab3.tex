\chapter{Analisis}
\label{chap:analisis}

\section{Masalah Definisi Pergerakan Kolektif Sebelumnya}
\label{sec:masalah-definisi-sebelumnya}

Seperti yang sudah dibahas pada bab sebelumnya, identifikasi pola pergerakan kolektif merupakan salah satu bagian penting dari analisis terhadap data lintasan. Secara informal, pergerakan kolektif merupakan sebuah keadaan di mana terdapat dua atau lebih entitas yang bergerak bersama selama kurun waktu tertentu. Dalam sebuah pergerakan kolektif, sebuah entitas dapat dikatakan bergerak bersama apabila lintasan yang ditempuh oleh entitas tersebut memiliki lintasan yang mirip dengan anggota-anggota lain yang tergabung dalam pergerakan kolektif tersebut.

Telah terdapat berbagai macam definisi formal yang dapat digunakan untuk mendeskripsikan pola pergerakan kolektif seperti \textit{flock} \cite{cao:flock, gudmundsson:flock}, \textit{convoy} \cite{jeung:convoys}, \textit{swarm} \cite{li:swarm}, \textit{group} \cite{buchin:group, yida:group}, dan masih banyak lagi. Seluruh definisi-definisi formal tersebut bergantung pada setidaknya pada tiga ukuran lintasan berikut untuk mengidentifikasi pergerakan kolektif:

\begin{itemize}
    \item \textbf{\textit{Size}}
    
    Jumlah entitas minimum yang harus bergerak bersama selama rentang waktu tertentu sehingga dapat teridentifikasi sebagai sebuah pergerakan kolektif.
    
    \item \textbf{Kedekatan Spasial}
    
    Batas jarak maksimum antara entitas-entitas yang bergerak agar dapat dianggap bergerak bersama. Ukuran ini sangat bergantung pada dimensi ruang gerak entitas.
    
    \item \textbf{Durasi Temporal}
    
    Durasi waktu minimum di mana entitas-entitas yang dianggap bergerak bersama harus dekat secara spasial sepanjang durasi waktu tersebut. 
\end{itemize}

Namun, definisi-definisi formal pergerakan kolektif yang sudah ada dapat menghasilkan hasil identifikasi pergerakan kolektif yang keliru atau kurang akurat pada dua kasus pergerakan yang lazim terjadi di dunia nyata. Hal tersebut mendorong perlunya pembuatan sebuah definisi formal pergerakan kolektif baru yang mampu mengatasi masalah-masalah yang terdapat pada definisi pergerakan kolektif sebelumnya. Kedua kasus tersebut beserta solusi yang diambil untuk mengatasi masalah akan dibahas melalui subbab berikut.

\subsection{Perbedaan Arah}
\label{subsec:beda-arah}
    
Berdasarkan syarat-syarat terjadinya pergerakan kolektif, sebuah entitas akan tergabung dalam suatu kelompok pergerakan kolektif apabila entitas tersebut dapat dikatakan dekat secara spasial dengan anggota-anggota dari pergerakan kolektif selama kurun waktu tertentu. Konsep tersebut sesuai dengan cara manusia mengidentifikasi pergerakan bersama pada sebuah data, di mana pergerakan bersama akan terjadi apabila terdapat sejumlah entitas di dalam pergerakan kolektif di mana anggota-anggotanya memiliki jarak yang cukup dekat selama durasi waktu yang cukup lama.

Namun, upaya identifikasi pergerakan kolektif yang hanya bergantung pada tiga ukuran lintasan yang telah disebutkan di atas dapat menimbulkan kesalahan identifikasi pada kasus di mana terdapat sebuah yang entitas bergerak berpapasan dengan entitas-entitas lainnya yang bergerak berlawanan arah dalam jarak yang cukup dekat dan durasi waktu yang cukup lama. Pergerakan tersebut dapat menyebabkan entitas yang bergerak berpapasan membentuk atau tergabung dalam sebuah pergerakan kolektif.
    
Dalam sebuah kasus yang ditunjukkan melalui gambar \ref{bab3:masalah-arah}, diberikan himpunan entitas yang bergerak $\mathcal{X}$ yang terdiri dari 4 buah entitas, jumlah entitas minimum $m$ sebanyak $3$ entitas, durasi waktu minimum $I$ selama $3$ satuan waktu, dan jarak antar entitas sepanjang $r$ satuan jarak. Pada himpunan $\mathcal{X}$, telah terdapat sebuah pergerakan kolektif $A$ di mana para anggotanya ditunjukkan sebagai entitas berwarna merah. Selain entitas-entitas yang tergabung dalam pergerakan kolektif, terdapat sebuah entitas $b$ yang ditunjukkan sebagai entitas berwarna biru. Entitas $b$ bergerak berpapasan dengan anggota-anggota dari pergerakan kolektif $A$ dalam jarak yang cukup dekat dan durasi waktu yang cukup lama. Berdasarkan fakta tersebut, definisi formal pergerakan kolektif yang telah dibuat sebelumnya akan mengidentifikasi entitas $b$ sebagai anggota dari pergerakan kolektif $A$. Hasil identifikasi tersebut tentunya bertentangan dengan hasil identifikasi yang dilakukan oleh manusia, di mana entitas berwarna biru tidak akan tergabung sebagai anggota pergerakan kolektif $A$ karena memiliki perbedaan arah yang besar dengan anggota-anggota dari pergerakan kolektif $A$.

\begin{figure}[t]
    \centering
    \includegraphics[width=0.535\textwidth]{Gambar/bab3:beda-arah.pdf}
    \caption[Masalah identifikasi pada kasus perbedaan arah]{Upaya identifikasi pergerakan kolektif oleh definisi formal yang telah dibuat sebelumnya dapat menyebabkan entitas berwarna biru tergabung dalam pergerakan kolektif entitas-entitas berwarna merah, walaupun terdapat perbedaan arah yang besar antara kedua lintasan.}
    \label{bab3:masalah-arah}
\end{figure}
    
Melalui analisis singkat terhadap masalah tersebut, tampaknya masalah perbedaan arah dapat diselesaikan dengan mudah melalui pengurangan jarak maksimum $r$ atau menambah durasi interval waktu $I$. Namun, kedua upaya penyelesaian tersebut akan mempengaruhi proses identifikasi pergerakan kolektif secara keseluruhan. Mengurangi jarak maksimum antar entitas atau menambah durasi interval waktu dapat menyebabkan sebuah pergerakan bersama yang sebelumnya dianggap sebagai sebuah pergerakan kolektif menjadi sebuah pergerakan bersama biasa karena jarak antar entitas yang terlalu jauh atau durasi waktu yang tidak cukup lama. Oleh karena itu, penyelesaian masalah identifikasi pergerakan kolektif pada kasus perbedaan arah menuntut pemanfaatan ukuran lintasan lainnya yang belum pernah digunakan sebelumnya.
    
Masalah tersebut dapat diselesaikan dengan memanfaatkan ukuran arah yang terdapat pada data lintasan sebagai ukuran tambahan dalam proses identifikasi pergerakan kolektif. Pada contoh kasus yang digambarkan melalui gambar \ref{bab3:masalah-arah}, entitas $b$ yang ditunjukkan sebagai entitas berwarna biru memiliki perbedaan arah lintasan sebesar 180 derajat dengan seluruh entitas yang tergabung sebagai anggota dari pergerakan kolektif $A$ yang ditunjukkan sebagai entitas-entitas berwarna merah. Perbedaan arah yang sangat besar menyebabkan entitas $b$ tidak akan tergabung sebagai anggota dari pergerakan kolektif $A$ walaupun entitas $b$ walaupun entitas $b$ memiliki jarak yang cukup dekat dengan seluruh anggota pergerakan kolektif $A$ dalam durasi waktu yang cukup lama. Berdasarkan upaya penyelesaian masalah tersebut, dapat dibuat simpulan sementara bahwa \textbf{suatu entitas akan tergabung sebagai anggota dalam sebuah kelompok pergerakan kolektif apabila perbedaan arah lintasan entitas tersebut dengan masing-masing anggota pergerakan kolektif kurang lebih atau sama dengan $\vartheta$}.

\begin{figure}
    \centering
    \includegraphics[height=9.5cm]{Gambar/bab3:beda-arah-lokal.pdf}
    \caption[Masalah identifikasi pada penghitungan arah secara global]{Pada kasus di mana $m = 2$ dan $k = 3$, menghitung arah secara global dapat menyebabkan pergerakan kolektif tidak terbentuk sama sekali karena perbedaan arah akhir yang terlalu ekstrim (ditunjukkan melalui garis putus-putus hijau)}
    \label{bab3:beda-arah-lokal}
\end{figure}

Hal penting yang tidak boleh dilupakan ketika menggunakan arah lintasan adalah cara menghitung arah dari sebuah lintasan. Secara umum, terdapat dua cara pandang untuk menghitung arah lintasan. Cara pandang pertama adalah dengan memandang arah lintasan secara global. Dalam cara pandang global, arah lintasan akan dihitung berdasarkan perbedaan arah pada awal dan akhir pergerakan entitas. Cara pandang tersebut dapat bermasalah ketika digunakan untuk mengukur kemiripan lintasan pada dua lintasan yang memiliki perbedaan arah akhir yang drastis. Dalam sebuah kasus yang ditunjukkan melalui gambar \ref{bab3:beda-arah-lokal}, diberikan dua buah entitas bergerak $a$ yang ditunjukkan sebagai entitas berwarna merah dan $b$ yang ditunjukkan sebagai entitas berwarna biru, jumlah entitas minimum $m$ sebanyak $2$, dan durasi waktu minimum $I$ selama $3$ satuan waktu. Apabila arah lintasan dipandang secara global, maka entitas $a$ dan entitas $b$ tidak dapat membentuk sebuah pergerakan kolektif karena memiliki perbedaan arah yang ekstrim. Hasil identifikasi tersebut bertentangan dengan hasil identifikasi pergerakan kolektif yang dilakukan oleh manusia, di mana entitas $a$ dan $b$ membentuk sebuah pergerakan kolektif selama rentang waktu $t_1$ -- $t_3$.

Masalah tersebut dapat diselesaikan dengan menggunakan cara pandang kedua, yaitu dengan memandang arah lintasan secara lokal. Dalam cara pandang lokal, perbedaan arah antara dua buah lintasan perbedaan dihitung pada setiap durasi waktu minimum $I$. Sebagai contoh pada kasus di mana panjang lintasan yang tercatat adalah sebesar $8$ dan durasi waktu minimum $I$ adalah sepanjang $3$ satuan waktu, maka perbedaan arah akan dihitung pada interval waktu $(t_1 - t_3), (t_2 - t_4), \ldots, (t_6 - t_8)$. Dalam kasus perbedaan arah yang ditunjukkan melalui gambar \ref{bab3:beda-arah-lokal}, cara pandang lokal menyebabkan entitas $a$ dan $b$ membentuk pergerakan kolektif pada interval waktu $t_1$ -- $t_3$, sesuai dengan hasil identifikasi yang dilakukan oleh manusia.

Berdasarkan hasil pengamatan tersebut, maka simpulan sebelumnya dapat dipertajam menjadi: \textbf{suatu entitas akan tergabung dalam sebuah kelompok pergerakan kolektif apabila perbedaan arah lintasan entitas tersebut dengan anggota-anggota pergerakan kolektif tidak melebihi $\vartheta$ yang dihitung pada setiap interval waktu minimum untuk membentuk sebuah pergerakan kolektif}.

Salah satu cara yang dapat digunakan untuk mengukur perbedaan arah antar dua buah lintasan adalah menggunakan \textit{cosine similarity} yang dapat dinyatakan sebagai:

\begin{equation}
    \cos (X, Y)= \frac{x \cdot y}{\|x\| \|y\|} = \frac{ \sum_{i=1}^{n}{x_i y_i}}{ \sqrt{\sum_{i=1}^{n}{(x_i)^2}} \sqrt{\sum_{i=1}^{n}{( y_i)^2}} }
    \label{bab3:cosine-similarity}
\end{equation}

Nilai \textit{cosine similarity} memiliki rentang nilai $[-1, 1]$, di mana \textit{cosine similarity} yang bernilai $-1$ menandakan bahwa kedua lintasan memiliki arah lintasan yang bertolak belakang dan \textit{cosine similarity} yang bernilai $1$ menandakan bahwa kedua lintasan memiliki arah lintasan yang sama persis. Penggunaan \textit{cosine similarity} sebagai ukuran perbedaan arah lintasan akan mempertajam simpulan sebelumnya menjadi: \textbf{suatu entitas akan tergabung dalam sebuah kelompok pergerakan kolektif apabila nilai \textit{cosine similarity} dari entitas tersebut dengan setiap anggota pergerakan kolektif lebih besar atau sama dengan $\vartheta$ yang dihitung pada setiap interval waktu minimum untuk membentuk sebuah pergerakan kolektif}.
    
\subsection{Perbedaan Kecepatan}
\label{subsec:beda-kecepatan}
    
Kemiripan lintasan merupakan salah satu faktor penentu utama dalam identifikasi pergerakan kolektif. Berdasarkan cara-cara penentuan kemiripan lintasan yang sudah dibahas pada subbab \ref{sec:kemiripan}, sebuah entitas akan memiliki lintasan yang mirip dengan entitas lain apabila memiliki jarak yang dianggap cukup dekat dalam durasi interval waktu minimal tertentu. Definisi tersebut sudah cukup baik bila digunakan untuk mengidentifikasi pergerakan kolektif pada sebagian besar kasus pergerakan yang terjadi di dunia nyata.

Sayangnya, definisi kemiripan tersebut belum cukup baik apabila digunakan untuk mengidentifikasi pergerakan kolektif di mana terdapat sebuah entitas yang memiliki perbedaan kecepatan dengan entitas bergerak lainnya. Perbedaan kecepatan tersebut dapat menyebabkan entitas yang bergerak lebih cepat atau lambat memiliki perbedaan jarak yang terlalu besar dengan entitas bergerak lainnya, sehingga entitas tersebut dianggap tidak bergerak bersamaan dengan entitas lain. Hal tersebut menjadi penting dalam proses identifikasi pergerakan kolektif, mengingat definisi pergerakan kolektif selalu memiliki syarat jumlah entitas minimum agar sekumpulan entitas dapat dianggap sebagai sebuah pergerakan kolektif.
    
Dalam sebuah kasus yang digambarkan melalui gambar \ref{bab3:masalah-kecepatan}, diberikan sebuah pergerakan kolektif $\mathcal{X}$ yang terdiri dari $3$ anggota $a$, $b$, dan $c$ yang masing-masing digambarkan dengan warna biru, kuning, dan merah, jumlah entitas minimum $m$ sebanyak $3$ entitas, dan durasi waktu minimum selama $3$ satuan waktu. Entitas $a$ dan $b$ bergerak bersama dengan kecepatan yang konstan sejak semula, sedangkan entitas $c$ bergerak lebih cepat dan meninggalkan kedua entitas tersebut sampai pada titik tertentu. Entitas $c$ kemudian berhenti dan menunggu kedua entitas tersebut untuk menyusul pada suatu titik. Setelah kedua entitas tersebut berhasil menyusul, ketiga entitas tersebut melanjutkan bergerak bersama pada kecepatan yang konstan. Pada kasus tersebut, perbedaan kecepatan antara entitas $c$ dan dua entitas lainnya tentunya akan menyebabkan bertambahnya perbedaan jarak antar entitas, di mana hal tersebut dapat menyebabkan entitas $c$ tidak memenuhi syarat perbedaan jarak dan durasi waktu minimum agar teridentifikasi sebagai anggota suatu pergerakan kolektif. Hasil identifikasi tersebut tidak sesuai dengan hasil identifikasi yang dilakukan oleh manusia, di mana entitas $c$ tetap akan tergabung dalam kelompok pergerakan kolektif bersama entitas $a$ dan $b$, walaupun memiliki entitas $c$ memiliki perbedaan jarak yang cukup jauh dengan entitas $a$ dan $b$ selama beberapa saat.

\begin{figure}[t]
    \centering
    \includegraphics[width=0.75\textwidth]{Gambar/bab3:beda-kecepatan.pdf}
    \caption[Masalah identifikasi pada kasus perbedaan kecepatan]{Pada kasus di mana $m = 3$ dan $t = 3$, perbedaan kecepatan entitas berwarna biru dapat berujung pada perbedaan jarak yang jauh dengan entitas-entitas merah sehingga pergerakan kolektif tidak terbentuk, walaupun entitas tersebut hanya memiliki perbedaan jarak yang jauh dengan entitas lain selama beberapa saat}
    \label{bab3:masalah-kecepatan}
\end{figure}
    
Melalui analisis singkat terhadap masalah tersebut, tampaknya masalah perbedaan kecepatan dapat diselesaikan dengan menambah jarak maksimum antar entitas sehingga perbedaan jarak antar entitas yang disebabkan oleh perbedaan kecepatan tidak mempengaruhi proses identifikasi pergerakan kolektif. Namun solusi tersebut dapat menghasilkan hasil identifikasi yang sama sekali tidak sesuai dengan definisi pergerakan bersama pada umumnya, di mana entitas-entitas yang bergerak bersama memiliki jarak yang berdekatan. Di sisi lain, kasus perbedaan kecepatan tidak dapat ditangani dengan baik oleh ukuran-ukuran yang biasanya terdapat dalam data lintasan. Oleh karena itu, penyelesaian masalah identifikasi pergerakan kolektif pada kasus perbedaan kecepatan menuntut penggunaan cara penghitungan kemiripan lintasan baru yang belum pernah digunakan oleh definisi pergerakan kolektif sebelumnya.

Masalah tersebut dapat diatasi dengan menggunakan algoritma \textit{dynamic time warping} untuk mengukur kemiripan antara dua buah lintasan. Seperti yang sudah dibahas pada subbab \ref{sec:dtw}, algoritma \textit{dynamic time warping} dapat digunakan untuk menghitung kemiripan lintasan dua buah entitas yangmemiliki kecepatan yang bervariasi atau mengalami perubahan kecepatan dengan frekuensi tertentu selama proses pencatatan data lintasan. Pada skripsi ini, algoritma \textit{dynamic time warping} akan digunakan untuk mengukur kemiripan dua buah lintasan pada setiap interval waktu minimum untuk membentuk sebuah pergerakan kolektif. Berdasarkan upaya penyelesaian tersebut, dapat diambil simpulan bahwa \textbf{suatu lintasan dari entitas yang bergerak akan dianggap mirip dengan lintasan dari entitas bergerak lainnya pada interval waktu $I$ apabila jarak \textit{dynamic time warping} antara kedua lintasan tersebut lebih kecil atau sama dengan $r$ yang dihitung selama periode waktu $I$}. 

\section{Definisi Pergerakan Kolektif Baru}

Berdasarkan pembahasan mengenai pergerakan kolektif pada subbab \ref{sec:collective-movement}, terdapat 3 aspek penting yang harus dipertimbangkan dalam membuat sebuah definisi pergerakan kolektif:

\begin{enumerate}
    \item \textbf{\textit{Size}} --- Berapa jumlah entitas minimum yang tergabung dalam sebuah himpunan entitas bergerak agar dapat dianggap sebagai sebuah pergerakan kolektif?
    \item \textbf{Kedekatan Spasial} --- Bagaimana cara mengukur kedekatan spasial antara dua buah entitas yang bergerak?
    \item \textbf{Durasi Temporal} --- Berapa durasi waktu minimum pergerakan bersama terjadi agar dapat dianggap sebagai sebuah pergerakan kolektif?
\end{enumerate}

Seluruh aspek yang telah disebutkan di atas merupakan sebuah parameter masukan yang ditentukan oleh peneliti sebelum proses identifikasi pergerakan kolektif dilakukan, sehingga sebuah definisi pergerakan kolektif yang sama dapat menghasilkan hasil identifikasi yang berbeda-beda yang dipengaruhi oleh parameter-parameter yang dimasukkan. 
Dengan mempertimbangkan aspek-aspek di atas, dapat dibuat sebuah definisi pergerakan kolektif baru bernama rombongan yang secara formal dinyatakan sebagai:

\noindent \textbf{\pergerakankolektif($m$, $k$, $r$, $\vartheta$)}. Diberikan sebuah himpunan entitas bergerak $\mathcal{X}$ yang memiliki jumlah entitas sebanyak $n$, jumlah entitas minimum sebanyak $m$ buah entitas di mana $k \in [2, n]$, interval waktu minimum selama $k$ satuan waktu di mana $k \geq 2$, jarak maksimum antar entitas sepanjang $r$ satuan panjang di mana $r \in \mathbb{R}^+$, dan nilai kemiripan sudut minimum sebesar $\vartheta$ di mana $\vartheta \in [-1, 1]$. Sebuah entitas bergerak $a \in \mathcal{X}$ dikatakan terhubung dengan entitas bergerak $b \in \mathcal{X}, a \neq b$ sepanjang interval waktu $I$ apabila jarak \textit{dynamic time warping} dari kedua entitas tersebut selama interval waktu $I$ lebih kecil atau sama dengan $r$ dan nilai \textit{cosine similarity} dari kedua entitas tersebut pada interval waktu $I$ lebih besar atau sama dengan $\vartheta$. Sebuah rombongan pada interval waktu $I$, di mana $I \geq t$, merupakan sebuah sub-himpunan $\mathcal{G} \in \mathcal{X}$ yang memiliki setidaknya $m$ buah entitas dan setiap anggotanya terhubung satu sama lain selama interval waktu $I$ secara konsekutif.

\section{Algoritma Identifikasi Rombongan}
\label{sec:algoritma}
    
Diberikan himpunan entitas yang bergerak $\mathcal{X}$, jumlah entitas minimum sebanyak $m$ buah entitas, interval waktu minimum selama $k$ satuan waktu, jarak \textit{dynamic time warping} maksimum sepanjang $r$ satuan panjang, dan nilai \textit{cosine similarity} minimum sebesar $\vartheta$. Himpunan rombongan yang terbentuk dalam himpunan entitas $\mathcal{X}$ dapat dicari menggunakan algoritma berikut:

\begin{algorithm}[h]
    \caption{Algoritma Identifikasi Rombongan}
    \SetKwInput{KwInput}{Input}                % Set the Input
    \SetKwInput{KwOutput}{Output}              % set the Output
    \DontPrintSemicolon
    
    \SetKwFunction{Kolektif}{FindKolektif}
 
    \SetKwProg{Fn}{Function}{:}{}
  
    \KwInput{
        \begin{itemize}
            \item Himpunan entitas bergerak $\mathcal{X}$
            \item Jumlah entitas minimum $m$
            \item Interval waktu minimum $k$
            \item Jarak \textit{dynamic time warping} maksimum antar entitas $r$
            \item Nilai \textit{cosine similarity} minimum $\vartheta$
        \end{itemize}
    }
    \KwOutput{Sebuah \textit{array} yang menyimpan himpunan dari himpunan rombongan yang teridentifikasi untuk setiap interval waktu yang memungkinkan}
    
    \Fn{\Kolektif{$\mathcal{X}, m, k, r, \vartheta$}}{
        $result \gets []$ \\
        
        \For{$(start, end)$ in each consecutive time intervals}{
            $collectiveMovements \gets Set()$ \\
            
            \For{each entity $a$ in $\mathcal{X}$}{
                $entitySet \gets [a]$ \\
                \For{each entity $b$ in $\mathcal{X}$}{
                    $isSimilar \gets TRUE$ \\
                    
                    \For{each entity $c$ in $entitySet$}{
                        $cSub \gets c[start \dots end]$ \\
                        $bSub \gets b[start \dots end]$ \\
                    
                        \If{$c \neq b$ AND $(DTWDistance(cSub, bSub) > r$ OR $CosineSimilarity(cSub, bSub) < \vartheta)$}{
                            $isSimilar \gets FALSE$
                        }
                    }
                    
                    \If{$isSimilar\;is\;equal\;to\;TRUE$}{
                        $Insert\;b\;into\;entitySet$
                    }
                    
                }
                
                \If{$length(entities) \geq m$}{
                    $Insert\;entitySet\;into\;collectiveMovement$
                }
            }
            $Insert\;collectiveMovements\;into\;result$
        }
    }
    \KwRet{$result$}
    
    \label{bab3:algoritma-identifikasi}
\end{algorithm}

\noindent Berikut merupakan uraian dari langkah-langkah yang dikerjakan oleh algoritma tersebut:

\begin{enumerate}
    \item Inisialisasi sebuah \textit{array} $\mathcal{R}$ yang akan menyimpan himpunan rombongan yang teridentifikasi untuk setiap interval waktu $I$.
    \item Cari setiap interval waktu konsekutif $I$ sepanjang $k$ satuan waktu yang dapat dibentuk. Interval-interval waktu konsekutif dapat dicari menggunakan teknik \textit{sliding window} \cite{ralf:03:sliding-window}. Sebagai contoh, pada himpunan entitas $\mathcal{X}$ sepanjang $5$ satuan waktu dan $k = 3$, maka interval waktu $I$ yang mungkin adalah $[1, 2, 3]$, $[2, 3, 4]$, dan $[3, 4, 5]$.
    \item Untuk setiap interval waktu konsekutif $I$ yang berdurasi $k$ satuan waktu:
    
    \begin{enumerate}
        \item Inisialiasi sebuah himpunan $\mathcal{T}$ yang akan menyimpan himpunan rombongan yang teridentifikasi pada interval waktu $I$.  
        \item Inisialisasi sebuah \textit{array} $\mathcal{S}$ untuk setiap entitas $a$ yang terdapat pada $\mathcal{X}$. Pada awalnya, himpunan $\mathcal{S}$ hanya akan beranggotakan entitas $a$.
        \item Hitung jarak \textit{dynamic time warping} dan nilai \textit{cosine similarity} dari setiap anggota himpunan $S$ dan setiap entitas lain pada himpunan $\mathcal{X}$ pada interval waktu $I$. Tambahkan setiap entitas yang memiliki jarak \textit{dynamic time warping} dengan setiap anggota himpunan $S$ yang lebih kecil sama dengan $r$ dan nilai \textit{cosine similiarity} dengan setiap anggota himpunan $S$ yang lebih besar sama dengan $\vartheta$ pada himpunan $S$.
        \item Tambahkan himpunan $\mathcal{S}$ pada \textit{array} $\mathcal{T}$ apabila jumlah anggota himpunan $\mathcal{S}$ lebih besar atau sama dengan $m$.
        \item Tambahkan \textit{array} $\mathcal{T}$ pada \textit{array} $\mathcal{R}$.
    \end{enumerate}
    \item Kembalikan \textit{array} $\mathcal{R}$ sebagai hasil. Pada titik ini, $\mathcal{R}$ akan berisi himpunan rombongan yang terbentuk pada setiap interval waktu $I$ yang berdurasi selama $k$ satuan waktu.
\end{enumerate}

Algoritma identifikasi memiliki kompleksitas waktu sebesar $O((t - k)n^3k^2)$ di mana $n$ merupakan jumlah entitas pada $\mathcal{X}$ dan $t$ merupakan panjang interval waktu terpanjang dari entitas anggota $\mathcal{X}$. Nilai tersebut didapatkan melalui perhitungan berikut:

\begin{enumerate}
    \item Pada baris $3$, algoritma akan memiliki kompleksitas waktu sebesar $O(t - k)$ di mana nilai tersebut diperoleh dengan menghitung jumlah interval waktu konsekutif yang mungkin dibuat. Jumlah interval waktu konsekutif yang dapat dibuat merupakan selisih dari panjang interval waktu terpanjang dari entitas anggota $\mathcal{X}$ dan durasi waktu konsekutif minimum untuk membentuk sebuah rombongan.
    \item Pada baris $5$, algoritma tersebut akan memiliki kompleksitas waktu sebesar $O((t - k)n)$ yang disebabkan oleh penghitungan seluruh entitas yang terdapat pada $\mathcal{X}$ yang memiliki kompleksitas sebesar $O(n)$.
    \item Pada baris $7$, algoritma tersebut akan memiliki kompleksitas waktu sebesar $O((t - k)n^2)$ yang disebabkan oleh penghitungan seluruh entitas yang terdapat pada $\mathcal{X}$ yang memiliki kompleksitas sebesar $O(n)$.
    \item Pada baris $9$, algoritma tersebut akan memiliki kompleksitas waktu sebesar $O((t - k)n^3)$ yang disebabkan oleh penghitungan seluruh entitas yang terdapat pada $\mathcal{S}$ yang memiliki kompleksitas sebesar $O(n)$.
    \item Pada baris $12$, algoritma tersebut akan memiliki kompleksitas waktu sebesar $O((t - k)n^3k^2)$ yang disebabkan oleh perhitungan algoritma \textit{dynamic time warping} yang memiliki kompleksitas waktu sebesar $O(k^2)$. Kompleksitas tersebut didapatkan dari hasil perkalian panjang dua lintasan yang dihitung yang masing-masing sebesar $k$. Pada titik ini, algoritma identifikasi akan memiliki nilai kompleksitas tertinggi selama algoritma dijalankan.
\end{enumerate}
