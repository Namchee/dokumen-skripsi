\chapter{Perancangan}
\label{chap:perancangan}

Bab ini akan membahas rancangan perangkat lunak identifikasi rombongan. Pada bab ini, akan dibahas mengenai analisis kebutuhan perangkat lunak serta arsitektur perangkat lunak berupa modul-modul perangkat lunak identifikasi rombongan yang akan dibuat.

\section{Fungsionalitas Perangkat Lunak}
\label{sec:functionality}

Melalui proses analisis terhadap kebutuhan perangkat lunak, perangkat lunak identifikasi rombongan yang akan dibuat membutuhkan tiga fungsionalitas berikut:

\begin{enumerate}
    \item Membaca dan menterjemahkan masukan dari pengguna berupa data pergerakan pejalan kaki dan parameter rombongan,
    \item Melakukan identifikasi rombongan-rombongan yang terbentuk dalam data pergerakan pejalan kaki yang diberikan dalam masukan pengguna.
    \item Melakukan evaluasi terhadap hasil identifikasi rombongan.
\end{enumerate}

Berdasarkan tiga kebutuhan tersebut, diputuskan bahwa perangkat lunak akan memiliki antarmuka pengguna dalam bentuk \textit{command line interface} sederhana. Interaksi antara pengguna dan perangkat lunak berdasarkan kedua fungsionalitas tersebut digambarkan melalui diagram konteks yang ditunjukkan melalui Gambar~\ref{bab4:context-diagram}.

\begin{figure}[htbp]
    \centering
    \includegraphics[width=0.7\textwidth]{Gambar/bab4:dfd-level-0.pdf}
    \caption{Diagram konteks}
    \label{bab4:context-diagram}
\end{figure}

\noindent Ketiga fungsionalitas utama dari perangkat lunak akan dipecah menjadi enam proses utama yaitu:

\begin{enumerate}
    \item \textbf{Penerjemahan Parameter Identifikasi}
    
    Pada proses penerjemahan parameter identifikasi, masukan-masukan dari pengguna yang diberikan melalui \textit{command line interface} akan divalidasi dan diterjemahkan menjadi bentuk data yang dapat dimengerti oleh perangkat lunak. Proses ini menerima masukan berupa masukan yang diberikan melalui \textit{command line interface} dan menghasilkan keluaran berupa parameter identifikasi rombongan.
    
    \item \textbf{Penerjemahan Data Pergerakan}
    
    Pada proses penerjemahan data pergerakan, data pergerakan yang akan dipakai pada proses identifikasi rombongan akan diterjemahkan menjadi bentuk data yang dapat dimengerti oleh perangkat lunak. Proses ini menerima masukan berupa data pergerakan dan menghasilkan keluaran dalam bentuk kumpulan entitas serta lintasan yang ditempuh oleh entitas tersebut pada data pergerakan yang digunakan.
    
    \item \textbf{Identifikasi Rombongan}
    
    Pada proses identifikasi rombongan, perangkat lunak akan melakukan identifikasi rombongan-rombongan dari data pergerakan yang sudah diterjemahkan pada proses penerjemahan data pergerakan menggunakan Algoritma~\ref{bab3:algoritma-identifikasi}.
    
    Proses ini menerima masukan berupa kumpulan entitas bergerak dan parameter identifikasi rombongan. Proses ini menghasilkan keluaran berupa kumpulan rombongan yang menyimpan informasi mengenai anggota-anggota rombongan serta durasi rombongan.
    
    \item \textbf{Penerjemahan Hasil Identifikasi Manusia}
    
    Sebelum hasil identifikasi rombongan dapat dievaluasi, perangkat lunak akan melakukan proses penerjemahan terhadap hasil identifikasi rombongan yang dihasilkan oleh manusia secara manual. Hasil identifikasi tersebut nantinya akan digunakan untuk membandingkan relevansi hasil identifikasi rombongan yang dihasilkan perangkat lunak dengan hasil identifikasi yang dilakukan oleh manusia. Proses ini menerima masukan berupa hasil identifikasi yang dilakukan oleh manusia dan menghasilkan keluaran dalam bentuk kumpulan rombongan.
    
    \item \textbf{Evaluasi Hasil Identifikasi}
    
    Pada proses evaluasi hasil identifikasi, hasil identifikasi yang dihasilkan oleh perangkat lunak akan diukur relevansinya dengan hasil identifikasi manusia yang dihasilkan melalui proses penerjemahan hasil identifikasi manusia. Relevansi hasil identifikasi akan diukur melalui nilai \textit{precision}, \textit{recall}, dan \textit{F1 score}.
    
    Proses ini menerima masukan berupa hasil identifikasi rombongan yang dihasilkan oleh perangkat lunak hasil identifikasi rombongan yang dilakukan oleh manusia secara manual. Proses ini menghasilkan keluaran berupa nilai relevansi dari hasil identifikasi rombongan serta meneruskan hasil identifikasi rombongan yang dilakukan oleh perangkat lunak.
    
    \item \textbf{Pembuatan Berkas Hasil}
    
    Setelah hasil identifikasi rombongan yang dihasilkan perangkat lunak melalui proses evaluasi, hasil identifikasi rombongan serta hasil evaluasi dari identifikasi rombongan akan disimpan dalam bentuk berkas hasil.
    
    Proses ini menerima masukan berupa nilai relevansi dan hasil identifikasi rombongan yang dilakukan oleh perangkat lunak. Proses ini menghasilkan keluaran berupa berkas teks yang mengandung informasi mengenai daftar rombongan yang teridentifikasi, parameter identifikasi yang digunakan, serta nilai relevansi dari hasil identifikasi tersebut.
\end{enumerate}

\noindent Hubungan antara keenam proses tersebut serta aliran data antara proses-proses yang berjalan ditunjukkan melalui diagram alir data tingkat pertama yang ditunjukkan melalui Gambar~\ref{bab4:dfd-level-1}.



\noindent Keenam proses tersebut akan diimplementasikan menjadi lima modul berikut:

\begin{enumerate}
    \item \textbf{Modul I/O}
    
    Modul I/O merupakan modul yang berfungsi untuk membaca masukan pengguna dan mencetak hasil identifikasi rombongan yang dihasilkan oleh modul rombongan. Terdapat dua proses yang akan dikerjakan oleh modul I/O, yaitu proses penerjemahan masukan pengguna dan proses pembuatan berkas hasil identifikasi rombongan.
    
    \item \textbf{Modul Penerjemah}
    
    Modul penerjemah merupakan modul yang berfungsi untuk menerjemahkan berkas masukan data pejalan kaki menjadi data yang dapat dipahami dan digunakan oleh perangkat lunak. Terdapat tiga proses yang akan dikerjakan oleh modul penerjemah, yaitu proses penerjemahan data pergerakan, proses penerjemahan parameter identifikasi, dan proses penerjemahan hasil identifikasi rombongan yang dilakukan oleh manusia.
    
    \item \textbf{Modul Rombongan}
    
    Modul rombongan merupakan modul yang berfungsi untuk mengidentifikasi kumpulan rombongan yang terbentuk dalam data pergerakan pejalan kaki yang sudah diterjemahkan oleh modul penerjemah. Terdapat satu proses yang akan dikerjakan oleh modul rombongan, yaitu proses identifikasi rombongan.
    
    \clearpage

    \item \textbf{Modul Spasial}
    
    Modul spasial merupakan modul yang berfungsi untuk melakukan perhitungan kedekatan spasial antar entitas yang nantinya akan digunakan oleh modul rombongan dalam proses identifikasi rombongan. Modul spasial akan memiliki fungsi untuk menghitung jarak \textit{dynamic time warping} dan nilai \textit{cosine similarity} yang akan digunakan pada proses identifikasi rombongan.
    
    Tujuan dari pemisahan modul rombongan dan modul spasial adalah untuk mempermudah proses validasi dari implementasi fungsi jarak \textit{dynamic time warping} dan nilai \textit{cosine similarity}.
    
    \item \textbf{Modul Evaluasi}
    
    Modul evaluasi merupakan modul yang berfungsi untuk melakukan evaluasi terhadap algoritma identifikasi rombongan yang dilakukan oleh modul rombongan. Evaluasi dilakukan dengan membandingkan hasil identifikasi rombongan dengan hasil identifikasi rombongan yang dilakukan oleh manusia pada data pergerakan yang sama. Terdapat satu proses yang dilakukan oleh modul evaluasi, yaitu evaluasi hasil identifikasi.
\end{enumerate}

\begin{figure}[t]
    \centering
    \includegraphics[width=\textwidth]{Gambar/bab4:dfd-level-1.pdf}
    \caption{Diagram alir data tingkat pertama}
    \label{bab4:dfd-level-1}
\end{figure}

\section{Spesifikasi Input dan Output}
\label{sec:input-output}

\subsection{Bentuk Data Rombongan}
\label{sec:input}

Data pergerakan pejalan kaki akan disimpan dalam sebuah berkas teks dengan format \texttt{.txt} dengan \textit{encoding} UTF-8. Berkas tersebut menyimpan informasi dari seluruh entitas yang tercatat dalam data pergerakan pejalan kaki. Berkas tersebut merupakan hasil ekstraksi dari informasi yang terdapat dalam sebuah video rekaman pergerakan pejalan kaki.

Informasi yang terdapat dalam data pejalan kaki disimpan dalam bentuk \textit{frame} yang telah diekstrak dari video rekaman pergerakan pejalan kaki. Sebuah \textit{frame} mengandung informasi dalam bentuk \texttt{<nomor-frame> <nomor-identitas> <posisi-x> <posisi-y>} yang ditulis dalam satu baris pada berkas teks. Nilai tambahan lainnya yang terdapat dalam berkas teks akan dihiraukan oleh perangkat lunak.

Nilai dari \texttt{<nomor-frame>} menunjukkan \textit{frame} pada video rekaman yang memiliki informasi yang terdapat pada baris teks. Nilai dari \texttt{<nomor-identitas>} menunjukkan nomor identitas unik dari entitas. Nilai dari \texttt{<posisi-x>} menunjukkan posisi entitas pada sumbu $X$ dari ruang gerak \textit{euclidean} $\mathbb{R}^2$. Hal yang sama juga berlaku untuk nilai \texttt{<posisi-y>} yang menyimpan posisi entitas pada sumbu $Y$ dari ruang gerak \textit{euclidean} $\mathbb{R}^2$.

\subsection{Bentuk Data Identifikasi Manusia}
\label{sec:human-identification}

Hasil identifikasi rombongan yang dilakukan secara manual oleh manusia akan disimpan dalam sebuah berkas teks dengan format \texttt{.txt} dengan \textit{encoding} UTF-8. Berkas tersebut akan menyimpan informasi mengenai seluruh rombongan yang diidentifikasi oleh manusia pada sebuah data pergerakan. Berkas hasil identifikasi manusia akan memiliki nama yang sama dengan nama berkas data pergerakan yang digunakan dalam proses identifikasi rombongan. 

Sebuah rombongan yang berhasil diidentifikasi secara manual oleh manusia direpresentasikan dalam satu baris dalam berkas teks. Setiap rombongan menyimpan informasi mengenai nomor identitas anggota yang tergabung dalam rombongan yang dimaksud. Data identifikasi manusia tidak menyimpan informasi mengenai durasi rombongan. 

\subsection{Bentuk Berkas Hasil}
\label{sec:result-text}

Berkas hasil akan disimpan dalam sebuah berkas teks dengan format \texttt{.txt} dengan \textit{encoding} UTF-8. Terdapat tiga buah informasi yang akan disimpan pada berkas hasil:

\begin{enumerate}
    \item Kumpulan parameter identifikasi yang digunakan pada proses identifikasi rombongan.
    \item Nilai relevansi hasil identifikasi rombongan dengan hasil identifikasi manusia yang direpresentasikan melalui nilai \textit{precision}, \textit{recall}, dan \textit{F1 score}.
    \item Kumpulan rombongan yang berhasil teridentifikasi oleh perangkat lunak.
\end{enumerate}

Sebuah rombongan yang berhasil teridentifikasi akan disimpan dalam satu baris berkas teks. Setiap rombongan akan disimpan dalam bentuk \texttt{<anggota>, <durasi>}. \texttt{<anggota>} menunjukkan kumpulan nomor identitas dari anggota rombongan yang dipisahkan menggunakan spasi. \texttt{<durasi>} menunjukkan kumpulan interval \textit{frame} konsekutif di mana rombongan tersebut terbentuk. Sebuah interval \textit{frame} ditulis sebagai sepasang \textit{frame} yang menunjukkan \textit{frame} mulai dan \textit{frame} berakhirnya rombongan yang dipisahkan menggunakan karakter strip (-).

Untuk mempermudah proses pengujian, berkas yang dihasilkan akan memiliki nama dalam bentuk \texttt{<data-pergerakan>-<waktu>}. Penyematan waktu pada nama berkas bertujuan agar hasil identifikasi rombongan pada data pergerakan yang sama dapat disimpan secara terpisah sehingga mempermudah proses analisis hasil identifikasi.

\section{Modul I/O}
\label{sec:des-io}

Modul I/O bertanggung jawab untuk melaksanakan dua proses berikut:

\begin{enumerate}
    \item \textbf{Input / Penerjemahan Masukan}
    
    Saat perangkat lunak dieksekusi melalui \textit{command line interface}, perangkat lunak akan meminta seperangkat masukan dari pengguna dalam bentuk \texttt{<nama-sumber-data> <m> <k> <r> <$\vartheta$> <n>}. $m$ merupakan jumlah anggota minimum dalam rombongan. $k$ merupakan rentang waktu minimum rombongan dalam satuan detik. $r$ merupakan jarak \textit{dynamic time warping} maksimum antar anggota rombongan dalam satuan meter. $\vartheta$ merupakan beda sudut maksimum antar anggota rombongan dalam satuan derajat. $n$ merupakan jumlah kedekatan minimum agar sebuah entitas dapat tergabung dalam sebuah rombongan.
    
    Sebelum perangkat lunak menjalankan proses selanjutnya, masukan tersebut akan divalidasi supaya memenuhi seluruh syarat berikut:
    
    \begin{enumerate}
        \item Nilai dari \texttt{m} harus merupakan bilangan bulat yang lebih besar dari 2.
        \item Nilai dari \texttt{k} harus merupakan bilangan positif.
        \item Nilai dari \texttt{r} harus merupakan bilangan positif.
        \item Nilai dari \texttt{n} harus merupakan bilangan non-negatif.
    \end{enumerate}
    
    Apabila masukan yang diberikan oleh pengguna tidak memenuhi seluruh syarat di atas, proses penerjemahan masukan akan menampilkan pesan kesalahan pada pengguna dan menghentikan proses eksekusi perangkat lunak.
    
    Setelah proses validasi masukan selesai, modul I/O akan mengembalikan masukan pengguna dalam bentuk tipe data komposit tanpa melakukan proses tambahan lainnya. Gambar~\ref{bab4:flowchart-input} menunjukkan diagram alir dari proses validasi masukan yang diberikan oleh pengguna.
    
    Selain masukan-masukan wajib di atas, modul ini mampu menerima dua masukan tambahan dari pengguna berupa \texttt{path} dan \texttt{redundant} yang dapat diberikan menggunakan \textit{flag}. Masukan \texttt{path} merupakan menentukan direktori tempat data pergerakan disimpan. Masukan \texttt{redundant} menentukan penggunaan algoritma pengurangan redundansi rombongan setelah algoritma identifikasi rombongan selesai melakukan identifikasi.
    
    \begin{figure}[htbp]
        \centering
        \includegraphics[width=\textwidth]{Gambar/bab4:flowchart-input.pdf}
        \caption{Diagram alir proses validasi masukan}
        \label{bab4:flowchart-input}
    \end{figure}
    
    \item \textbf{Output / Pembuatan Berkas Hasil}
    
    Setelah proses identifikasi rombongan selesai, hasil dari proses identifikasi tersebut akan diteruskan pada modul I/O agar hasil tersebut dapat disimpan dalam bentuk berkas teks biasa yang mudah dibaca dan dievaluasi.
    
    Berkas teks hasil akan mengandung informasi mengenai parameter identifikasi rombongan yang digunakan, evaluasi hasil identifikasi rombongan yang dihasilkan oleh modul evaluasi, serta daftar rombongan yang berhasil teridentifikasi.
\end{enumerate}

\section{Modul Penerjemah}
\label{sec:des-parser}

Modul penerjemah bertanggung jawab untuk melaksanakan tiga proses berikut:

\begin{enumerate}
    \item \textbf{Penerjemahan Data Pergerakan}
    
    Setelah nama berkas  modul I/O, modul penerjemah akan mengubah data pergerakan yang terdapat pada berkas data menjadi bentuk yang dapat dipahami oleh perangkat lunak. Berikut merupakan langkah-langkah yang dikerjakan dalam proses penerjemahan data pergerakan:
    
    \begin{enumerate}
        \item Menguji validitas berkas data pergerakan pejalan kaki. Apabila berkas tidak dapat dibaca oleh perangkat lunak, maka modul penerjemah akan menampilkan pesan kesalahan dan menghentikan proses eksekusi perangkat lunak.
        
        \item Membaca setiap baris teks yang tercatat dalam berkas data pergerakan.
    
        \item Untuk setiap baris data yang berhasil dibaca oleh perangkat lunak, simpan informasi posisi entitas dan nilai \textit{frame} yang menyimpan informasi posisi tersebut.
        
        \item Untuk setiap entitas yang berhasil dibaca, periksa keberadaan entitas tersebut untuk setiap \textit{frame} yang tercatat. Apabila entitas tidak terdapat pada sebuah \textit{frame}, tandai posisi entitas tersebut pada \textit{frame} yang dimaksud menggunakan sebuah \textit{anchor value}. \textit{Anchor value} akan digunakan untuk memeriksa keberadaan entitas pada sebuah \textit{frame}.
        
        \item Kembalikan setiap informasi entitas berupa posisi setiap entitas bergerak yang tercatat dalam data pergerakan pada setiap \textit{frame} yang terdapat dalam data pergerakan.
    \end{enumerate}
    
    Gambar~\ref{bab4:flowchart-penerjemah} menunjukkan diagram alir dari proses penerjemahan data pergerakan yang berlangsung dalam modul penerjemah.
    
    \begin{figure}[htbp]
        \centering
        \includegraphics[width=\textwidth]{Gambar/bab4:flowchart-penerjemah.pdf}
        \caption{Diagram alir proses penerjemahan data pergerakan}
        \label{bab4:flowchart-penerjemah}
    \end{figure}

    \item \textbf{Penerjemahan Parameter Identifikasi}
    
    Sebelum proses identifikasi rombongan dilakukan, perangkat lunak perlu mengetahui parameter-parameter yang akan digunakan dalam proses identifikasi rombongan. Parameter identifikasi diperoleh dengan mengolah masukan pengguna yang diperoleh dari modul I/O.
    
    Terdapat dua nilai yang akan diproses oleh fungsi ini, yaitu $k$ dan $\vartheta$. Nilai $k$ akan diterjemahkan menjadi satuan \textit{frame}. Nilai dari $\vartheta$ akan diterjemahkan menjadi nilai \textit{cosinus} dari sudut tersebut. Parameter-parameter lainnya tidak akan melalui proses lanjutan apapun dan akan langsung diteruskan pada modul rombongan.
    
    \item \textbf{Penerjemahan Hasil Identifikasi Manusia}
    
    Sebelum hasil identifikasi rombongan pada data pejalan kaki diuji melalui pengujian kualitatif dan kuantitatif yang dilakukan oleh modul evaluasi, perangkat lunak memerlukan hasil identifikasi manusia yang dijadikan sebagai acuan. Data hasil identifikasi acuan yang disimpan dalam berkas teks akan diterjemahkan menjadi bentuk yang dapat dimengerti oleh perangkat lunak sehingga dapat digunakan pada proses evaluasi hasil identifikasi rombongan. Gambar~\ref{bab4:flowchart-manusia} menunjukkan diagram alir dari proses penerjemahan hasil identifikasi manusia.
    
    \begin{figure}[htbp]
        \centering
        \includegraphics[width=\textwidth]{Gambar/bab4:flowchart-manusia.pdf}
        \caption{Diagram alir proses penerjemahan hasil identifikasi manusia}
        \label{bab4:flowchart-manusia}
    \end{figure}
\end{enumerate}

\section{Modul Rombongan}
\label{sec:des-rombongan}

Modul rombongan merupakan modul yang berfungsi sebagai implementasi konkrit dari algoritma identifikasi rombongan yang ditunjukkan melalui Algoritma~\ref{bab3:algoritma-identifikasi}. Modul rombongan akan menghasilkan rombongan-rombongan yang teridentifikasi pada data pergerakan yang sesuai dengan parameter identifikasi yang diterjemahkan dari masukan pengguna oleh modul penerjemah. Pada proses identifikasi rombongan, modul rombongan akan menggunakan algoritma \textit{dynamic time warping} dan perhitungan \textit{cosine similarity} yang diimplementasikan dalam modul spasial.

\section{Modul Spasial}
\label{sec:des-spatial}

Modul spasial merupakan modul yang berisi implementasi dari algoritma-algoritma yang akan digunakan untuk menentukan kedekatan spasial antara dua buah entitas yang digunakan selama proses identifikasi rombongan. Implementasi tersebut dipisahkan dari modul rombongan untuk mempermudah pengujian perangkat lunak. Modul spasial akan mengimplementasikan algoritma \textit{dynamic time warping} yang ditunjukkan oleh Algoritma~\ref{bab2:dtw-pseudocode} dan perhitungan \textit{cosine similarity} yang ditunjukkan oleh Rumus~\ref{bab2:cosine-similarity}.

\section{Modul Evaluasi}
\label{sec:des-evaluation}

Modul evaluasi merupakan modul yang berfungsi untuk melakukan evaluasi hasil identifikasi rombongan yang dihasilkan oleh modul rombongan dengan membandingkan hasil identifikasi rombongan dengan data \textit{ground truth} yang sesuai dengan masukan data pergerakan. Evaluasi hasil identifikasi dilakukan secara kuantitatif dengan menghitung nilai \textit{precision}, \textit{recall}, dan \textit{F1 score} dari hasil identifikasi rombongan.

\section{Struktur Data}
\label{sec:des-struct}

Untuk mempermudah proses implementasi perangkat lunak, diperlukan beberapa struktur data komposit yang terdiri dari berbagai nilai primitif. Terdapat enam buah struktur data komposit yang akan dibuat, antara lain:

\begin{enumerate}
    \item \textbf{Arguments}
    
    Struktur data ini akan menyimpan masukan-masukan mentah dari pengguna yang diberikan melalui \textit{command line interface}. Masukan-masukan tersebut akan melalui proses validasi dasar dan diubah menjadi tipe data yang dapat dipahami oleh perangkat lunak. Struktur data ini akan dihasilkan oleh modul I/O.
    
    \item \textbf{Parameters}
    
    Struktur data ini akan menyimpan masukan-masukan pengguna yang telah diproses menjadi parameter-parameter yang akan digunakan oleh algoritma identifikasi rombongan. Struktur data ini akan dihasilkan oleh modul penerjemah.
    
    \item \textbf{Entity}
    
    Struktur data ini akan menyimpan informasi-informasi penting dari entitas yang tercatat dalam data pergerakan pejalan kaki seperti nomor identitas dari entitas serta seluruh posisi dari entitas tersebut pada ruang gerak $\mathbb{R}^2$. Struktur data ini akan dihasilkan oleh modul penerjemah.
    
    \item \textbf{MovementData}
    
    Struktur data ini akan menyimpan informasi mengenai rombongan-rombongan yang teridentifikasi pada hasil identifikasi rombongan yang dilakukan secara manual oleh manusia. Struktur data ini akan dihasilkan oleh modul penerjemah.
    
    \item \textbf{Rombongan}
    
    Struktur data ini akan menyimpan informasi mengenai nomor identitas dari entitas-entitas yang tergabung dalam rombongan serta daftar durasi terbentuknya rombongan. Durasi rombongan akan menyimpan nilai \textit{frame} pertama di mana sekelompok entitas yang bergerak bersama memenuhi syarat pembentukan rombongan, dan nilai \textit{frame} terakhir dimana anggota-anggota rombongan masih bergerak bersama sebagai rombongan sebelum pergerakan kolektif tersebut tidak lagi memenuhi syarat-syarat pembentukan rombongan. Struktur data ini akan dihasilkan oleh modul rombongan.
    
    \item \textbf{Score}
    
    Struktur data ini akan menyimpan nilai relevansi dari hasil identifikasi rombongan yang dilakukan oleh perangkat lunak dibandingkan dengan hasil identifikasi yang dilakukan secara manual oleh manusia. Struktur data ini akan dihasilkan oleh modul evaluasi.
\end{enumerate}