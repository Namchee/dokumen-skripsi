%versi 2 (8-10-2016) 
\chapter{Pendahuluan}
\label{chap:intro}
   
\section{Latar Belakang}
\label{sec:label}

Bergerak merupakan sebuah aktivitas yang selalu kita temui setiap saat. Berdasarkan Kamus Besar Bahasa Indonesia, bergerak merupakan sebuah kegiatan berpindah dari tempat atau kedudukan semula menuju tempat atau kedudukan lain\footnote{Badan Pengembangan dan Pembinaan Bahasa Kementrian Pendidikan dan Kebudayan, \textit{Bergerak}, diakses pada tanggal 23 Desember 2020, \url{https://kbbi.kemdikbud.go.id/entri/bergerak}.}. Segala sesuatu yang ada di sekitar kita selalu bergerak untuk mencapai tujuan tertentu seperti yang ditunjukkan melalui Gambar~\ref{bab1:pergerakan}. Manusia selalu bergerak untuk melakukan aktivitas sehari-hari seperti yang ditunjukkan melalui Gambar~\ref{bab1:manusia}. Berbagai jenis hewan melakukan migrasi untuk mencari lingkungan baru yang lebih mampu untuk menunjang kehidupan. Tanaman dapat bergerak melalui fototropisme yang menyebabkan tanaman bertumbuh mengikuti arah sinar matahari yang ditunjukkan melalui Gambar~\ref{bab1:sunflower}. Bahkan, bumi selalu berputar mengelilingi porosnya yang menyebabkan pergantian hari. Dapat dikatakan bahwa bergerak merupakan aktivitas yang tak dapat dilepaskan dari kehidupan sehari-hari. Hal tersebut menumbuhkan rasa ketertarikan manusia untuk mengumpulkan, menyelidiki, serta mempelajari segala aspek mengenai pergerakan. Sayangnya, keterbatasan cara mendapatkan data pergerakan menghambat penelitian mengenai pergerakan di masa lalu.

\begin{figure}[h]
    \centering
    \begin{subfigure}[b]{0.425\textwidth}
        \includegraphics[width=\textwidth, height=4.5cm]{Gambar/bab1:manusia.jpg}
        \caption{Manusia bergerak untuk memenuhi kebutuhan sehari-hari\protect\footnotemark[2]}
        \label{bab1:manusia}
    \end{subfigure} \hspace{0.35cm}
    \begin{subfigure}[b]{0.425\textwidth}
        \includegraphics[width=\textwidth, height=4.5cm]{Gambar/bab1:sunflower.jpg}
        \caption{Bunga matahari bertumbuh mengikuti arah sinar matahari\protect\footnotemark[3]}
        \label{bab1:sunflower}
    \end{subfigure}
    \caption[Aktivitas pergerakan]{
    Aktivitas pergerakan yang dilakukan oleh berbagai entitas di sekitar kita}
    \label{bab1:pergerakan}
\end{figure}

\footnotetext[2]{Alvin Mamudov,  2017, diakses pada tanggal 4 Januari 2021, \url{https://unsplash.com/photos/FlLHbmF3AHc}.}

\footnotetext[3]{Lisa Pellegrini, 2016, diakses pada tanggal 4 Januari 2021, \url{https://unsplash.com/photos/XCvy_eufErI}.}

Kemunculan sistem informasi seperti \textit{Geographic Information System} (GIS) menyebabkan data pergerakan yang bersumber dari berbagai entitas menjadi semakin banyak, mudah didapatkan, dan mudah dikelola. Hal tersebut memicu kembalinya pertumbuhan minat penelitian mengenai pergerakan oleh berbagai peneliti dalam berbagai bidang. Terdapat banyak hal menarik yang dapat dimanfaatkan melalui analisis terhadap data pergerakan. Salah satu contoh nyata dari pemanfaatan analisis data pergerakan adalah analisis pergerakan bebek domestik, di mana melalui analisis terhadap data pergerakan bebek, virolog dapat mengidentifikasi daerah-daerah yang memiliki risiko persebaran flu burung yang tinggi~\cite{diann:01:movement-analysis}.

\begin{figure}[h]
    \centering
    \begin{subfigure}[h]{0.45\textwidth}
        \centering
        \includegraphics[width=\textwidth]{Gambar/bab1:pesawat.jpg}
        \caption{Lintasan yang ditempuh oleh pesawat tempur militer\protect\footnotemark[4]}
        \label{bab1:fighter-plane}
    \end{subfigure} \hspace{0.25cm}
    \begin{subfigure}[h]{0.45\textwidth}
        \centering
        \includegraphics[width=\textwidth]{Gambar/bab1:roket.jpg}
        \caption{Lintasan yang ditempuh oleh sebuah roket uji coba\protect\footnotemark[5]}
        \label{bab1:rocket}
    \end{subfigure}
    \caption[Lintasan dalam dunia nyata]{Lintasan yang dapat kita temui di dunia nyata}
    \label{bab1:trajectory}
\end{figure}

Dalam sebuah pergerakan, setiap entitas yang bergerak memiliki lintasan (\textit{trajectory}) seperti yang ditunjukkan melalui Gambar~\ref{bab1:fighter-plane} dan Gambar~\ref{bab1:rocket}. Lintasan merupakan jalur yang dilalui oleh entitas selama melakukan pergerakan dalam rentang waktu tertentu. Data mengenai lintasan yang ditempuh oleh entitas yang bergerak bersifat kontinu. Data lintasan dapat diperoleh melalui berbagai cara, di mana hal tersebut ditentukan oleh tipe entitas, lingkungan tempat terjadinya pergerakan, teknologi yang digunakan, dan lain sebagainya~\cite{wiratma:trajectory}. Pada zaman dahulu, peneliti harus mengambil data pergerakan dengan cara manual seperti melakukan penelusuran pergerakan pada kura-kura secara manual~\cite{stickel:02:turtle} atau dengan menyematkan sebuah cincin pada burung bangau~\cite{velden:01:cranes}. Saat ini, data mengenai lintasan dapat diperoleh dengan mudah melalui pemanfaatan teknologi modern seperti sistem Argos-Doppler dan \textit{global positioning system} (GPS) yang lazim ditemukan pada telepon pintar, di mana keduanya sama-sama memanfaatkan teknologi satelit~\cite{carter:argos}. Kedua sistem tersebut memiliki beberapa kelebihan dibandingkan penggunaan cara-cara tradisional seperti memiliki akurasi yang lebih baik, mampu memperoleh lebih banyak data pergerakan, dapat dikustomisasi sesuai kebutuhan, dan masih banyak lagi.

\footnotetext[4]{Aral Tasher, \textit{Thunderbirds at Melbourne Airshow}, 2018, diakses pada tanggal 11 Juni 2021, \url{https://unsplash.com/photos/nsqDXJdYzF8}}

\footnotetext[5]{SpaceX, \textit{Ray of light near body of water}, 2016, diakses pada tanggal 11 Juni 2021, \url{https://unsplash.com/photos/-p-KCm6xB9I}}

Sayangnya, perkembangan teknologi tetap tidak mampu untuk merekam data lintasan secara sempurna. Hal tersebut menyebabkan data lintasan yang diperoleh tidak bersifat kontinu seperti yang diharapkan, melainkan bersifat diskrit. Berdasarkan sifat tersebut, data lintasan yang diperoleh akan direpresentasikan sebagai catatan posisi dari entitas yang bergerak yang diurutkan berdasarkan titik waktu. Secara formal, lintasan merupakan himpunan dari pasangan posisi-waktu $(p_0, t_0), (p_1, t_1), \ldots, (p_x, t_x)$, di mana $p_i$ merupakan posisi entitas yang relatif terhadap ruang gerak entitas pada titik waktu $t_i$. Pada umumnya, entitas akan bergerak dalam sebuah ruang \textit{euclidean} dua dimensi $\mathbb{R}^2$ atau tiga dimensi $\mathbb{R}^3$.

Setiap lintasan memiliki atribut-atribut dengan nilai tertentu sebagai karakteristik yang membedakan satu lintasan dengan lintasan lain. Atribut-atribut tersebut kemudian dapat diolah menjadi ukuran lintasan yang memiliki kegunaan yang lebih spesifik. Analisis terhadap data pergerakan selalu memanfaatkan setidaknya salah satu ukuran yang terdapat pada data lintasan. Sebagai contoh, kecepatan dapat digunakan untuk mengukur pengaruh angin pada pergerakan burung~\cite{safi:speed}.

\iffalse \lionov{jadiin paragraf baru, kan ini beda topik} \fi

Terdapat berbagai jenis analisis yang dapat dilakukan pada data pergerakan seperti segmentasi lintasan~\cite{mann:01:segmentation}, pengukuran kemiripan lintasan~\cite{rote:01:hausdorff, alt:01:frechet, muller:dtw}, \textit{clustering} pada entitas yang bergerak~\cite{lee:01:clustering}, dan identifikasi pergerakan kolektif yang menjadi fokus utama pada skripsi ini. Tujuan dari analisis pergerakan kolektif adalah mengidentifikasi kelompok pergerakan yang terbentuk dari entitas-entitas yang bergerak bersama dalam rentang waktu yang cukup lama. Identifikasi pergerakan kolektif memiliki pemanfaatan dalam berbagai bidang. Sebagai contoh, identifikasi pergerakan kolektif dapat dimanfaatkan pada bidang keamanan untuk mengidentifikasi pergerakan mencurigakan dari sekelompok orang~\cite{makris:01:security}. Gambar~\ref{bab1:koi} dan Gambar~\ref{bab1:ped} menunjukkan contoh pergerakan kolektif yang dapat ditemukan di dunia nyata.

\begin{figure}[h]
    \centering
    \begin{subfigure}[h]{0.45\textwidth}
        \centering
        \includegraphics[width=\textwidth, height=10.5cm]{Gambar/bab1:koi.jpg}
        \caption{Kumpulan ikan koi dalam kolam yang memiliki pergerakan kolektif\protect\footnotemark[6]}
        \label{bab1:koi}
    \end{subfigure} \hspace{0.25cm}
    \begin{subfigure}[h]{0.45\textwidth}
        \centering
        \includegraphics[width=\textwidth, height=10.5cm]{Gambar/bab1:ped.jpg}
        \caption{Lalu lintas pejalan kaki yang dibentuk oleh banyak kelompok pejalan kaki\protect\footnotemark[7]}
        \label{bab1:ped}
    \end{subfigure}
    \caption[Pergerakan kolektif dunia nyata]{Contoh pergerakan kolektif yang dapat kita temui di dunia nyata}
    \label{bab1:collective-movement}
\end{figure}

\footnotetext[6]{Elliot Andrews
, \textit{School of koi fish}, 2019, diakses pada tanggal 17 Juni 2021, \url{https://unsplash.com/photos/ZV8OoFZ1nRc}}

\footnotetext[7]{Jan Antonin Kolar, \textit{Aerial shot of pedestrians}, 2019, diakses pada tanggal 17 Juni 2021, \url{https://unsplash.com/photos/hN_zCni3ILg}}

\iffalse 

\lionov{sebaiknya ada ilustrasi kayak Figure 1.7 di thesis, buat memperjelas apa itu pergerakan kolektif} \cristopher{sedang digambar}.
\fi

Ada berbagai macam definisi formal yang sudah dibuat untuk mengidentifikasi pergerakan kolektif seperti \textit{flock}~\cite{cao:flock, gudmundsson:flock}, \textit{convoy}~\cite{jeung:convoys}, \textit{group}~\cite{buchin:group, yida:group}, dan masih banyak lagi. Seluruh definisi formal tersebut bergantung pada parameter \textit{size}\iffalse \lionov{ini jadi aneh, pake {\it size} aja deh} kelompok \fi, kedekatan spasial, dan durasi untuk melakukan identifikasi pergerakan kolektif pada sekelompok entitas yang bergerak bersama. \textit{Size} menentukan jumlah anggota minimum yang harus tergabung dalam sebuah pergerakan kolektif. Kedekatan spasial menentukan batas maksimum jarak antara anggota-anggota pergerakan kolektif. Durasi menentukan waktu minimum pergerakan bersama dari seluruh anggota pergerakan kolektif.

Namun, identifikasi pergerakan kolektif pada data pergerakan dunia nyata membawa tantangan tersendiri pada beberapa definisi pergerakan kolektif yang sudah ada. Tantangan tersebut disebabkan oleh adanya perilaku pergerakan yang tidak umum. Berdasarkan pengamatan yang dihasilkan oleh rekaman video hasil identifikasi yang diteliti sebelumnya oleh Wiratma, dkk\footnotemark[8], terdapat setidaknya dua masalah dalam mengidentifikasi pergerakan kolektif dalam data pergerakan dunia nyata. Masalah pertama adalah adanya perbedaan arah entitas dalam jarak dekat pada data pergerakan yang ditunjukkan melalui Gambar~\ref{bab1:beda-arah}.

\footnotetext[8]{Daftar lengkap rekaman video identifikasi pergerakan kolektif hasil penelitian yang dilakukan oleh Wiratma, dkk dapat dilihat melalui tautan \url{http://groupingvideos.epizy.com/}.}

Pada kedua contoh tersebut, dapat dilihat bahwa terdapat dua pergerakan kolektif yang bergerak berlawanan arah dalam jarak yang dianggap cukup dekat dalam durasi yang cukup lama. Kondisi tersebut menyebabkan definisi pergerakan kolektif yang diuji melakukan kesalahan identifikasi dengan mengikutsertakan entitas yang berpapasan berlawanan arah sebagai anggota pergerakan kolektif selama kedua pihak berpapasan. Hal tersebut disebabkan karena kondisi tersebut memenuhi syarat kedekatan spasial dan durasi dari pergerakan kolektif, di mana entitas yang bergerak akan memiliki jarak yang dekat dengan anggota pergerakan kolektif dalam rentang waktu yang cukup lama mengingat keduanya memiliki kecepatan yang rendah. Hasil tersebut tentunya bertentangan dengan hasil identifikasi yang dilakukan oleh manusia karena secara nalar, entitas yang berpapasan tersebut memiliki perbedaan arah yang besar dengan anggota-anggota pergerakan kolektif sehingga entitas tersebut tidak bisa diidentifikasi sebagai anggota pergerakan kolektif.

\begin{figure}[t]
    \centering
    \captionsetup{width=0.85\textwidth}
    \begin{subfigure}[h]{0.45\textwidth}
        \centering
        \includegraphics[height=4.75cm]{Gambar/bab1:beda-arah-1.pdf}
    \end{subfigure} \hspace{0.25cm}
    \begin{subfigure}[h]{0.45\textwidth}
        \centering
        \includegraphics[height=4.75cm]{Gambar/bab1:beda-arah-2.pdf}
    \end{subfigure}
    \caption[Masalah perbedaan arah pada identifikasi pergerakan kolektif]{Adanya perbedaan arah dalam waktu yang cukup lama dan jarak yang cukup dekat dapat menyebabkan entitas yang berlawanan arah tergabung sebagai anggota pergerakan kolektif}
    \label{bab1:beda-arah}
\end{figure}

\begin{figure}[t]
    \centering
    \captionsetup{width=0.75\textwidth}
    \begin{subfigure}[h]{0.25\textwidth}
        \centering
        \includegraphics[height=7.5cm]{Gambar/bab1:beda-kecepatan-1.pdf}
    \end{subfigure}
    \begin{subfigure}[h]{0.25\textwidth}
        \centering
        \includegraphics[height=7.5cm]{Gambar/bab1:beda-kecepatan-2.pdf}
    \end{subfigure}
    \begin{subfigure}[h]{0.25\textwidth}
        \centering
        \includegraphics[height=7.5cm]{Gambar/bab1:beda-kecepatan-3.pdf}
    \end{subfigure}
    \caption[Masalah perbedaan kecepatan pada identifikasi pergerakan kolektif]{Adanya perbedaan kecepatan antara anggota-anggota pergerakan kolektif dapat menyebabkan anggota tidak dekat secara spasial sehingga pergerakan kolektif tidak terbentuk}
    \label{bab1:beda-kecepatan}
\end{figure}

Masalah kedua adalah adanya perbedaan kecepatan antara anggota-anggota pergerakan kolektif yang ditunjukkan melalui Gambar~\ref{bab1:beda-kecepatan}. Pada gambar-gambar tersebut, terdapat dua buah entitas yang membentuk sebuah pergerakan kolektif. Satu anggota dari pergerakan kolektif bergerak memimpin pergerakan kolektif dengan berjalan mendahului anggota lain. Karena entitas tersebut terus memimpin berjalan dengan menatap ke depan, entitas tersebut secara tidak sadar memiliki kecepatan yang lebih cepat dan meninggalkan anggota lain. Kondisi tersebut menyebabkan entitas yang memimpin memiliki perbedaan jarak yang cukup jauh dengan anggota lain. Setelah beberapa saat, entitas yang memimpin berhenti kemudian menoleh ke belakang untuk memastikan apakah dirinya dan entitas lain masih bergerak secara kolektif. Karena perbedaan jarak yang jauh, entitas yang memimpin memutuskan untuk memperlambat gerakan dan menunggu entitas lainnya untuk menyusul sampai jarak tertentu kemudian menyesuaikan kecepatannya dengan entitas lain untuk kembali bergerak secara kolektif dengan anggota lain. Pada kasus tersebut, definisi-definisi formal pergerakan kolektif yang diamati menghasilkan identifikasi di mana kedua entitas tidak membentuk pergerakan kolektif sama sekali. Hal tersebut disebabkan karena entitas yang memimpin memiliki jarak yang jauh dengan anggota-anggota pergerakan kolektif lainnya selama beberapa saat sehingga syarat kedekatan spasial dan durasi tidak terpenuhi. Hasil tersebut bertentangan dengan hasil identifikasi pergerakan kolektif yang dilakukan oleh manusia, di mana entitas yang memimpin akan selalu tergabung sebagai anggota pergerakan kolektif mengingat kasus serupa lazim terjadi dalam sebuah rombongan pejalan kaki dunia nyata. Dua masalah tersebut mendorong perlunya perluasan ukuran lintasan yang digunakan dalam proses identifikasi terhadap sebuah pergerakan kolektif serta pembuatan definisi formal pergerakan kolektif baru yang mampu mengatasi masalah identifikasi pergerakan kolektif pada kasus-kasus serupa. 

Pada penelitian ini, akan dibuat sebuah definisi formal pergerakan kolektif baru yang akan memperluas ukuran-ukuran penentu yang digunakan untuk mengidentifikasi pergerakan kolektif. Setelah definisi formal selesai dibuat, akan dibuat sebuah algoritma yang mampu mengidentifikasi pergerakan kolektif sesuai dengan definisi formal pergerakan kolektif baru yang dibuat. Algoritma identifikasi akan diimplementasikan menjadi sebuah perangkat lunak menggunakan bahasa pemrograman C++. Setelah perangkat lunak selesai dibuat, akan dilakukan eksperimen pada data pergerakan dunia nyata untuk mengetahui hasil identifikasi pergerakan kolektif yang sesuai dengan definisi pergerakan kolektif baru yang dibuat dan hasilnya akan divisualiasikan dengan menggunakan sebuah model pewarnaan di dalam video rekaman data pejalan kaki. Selanjutnya, hasil identifikasi akan dievaluasi secara kuantitatif menggunakan nilai \textit{precision}, \textit{recall}, dan \textit{F1 score} yang dibandingkan dengan hasil identifikasi pergerakan kolektif yang dilakukan oleh manusia dan secara kualitatif melalui analisis terhadap hasil visualisasi pada rekaman video pergerakan pejalan kaki. Visualisasi dilakukan menggunakan bantuan perangkat lunak lain yang dikembangkan oleh Wiratma, dkk~\cite{wiratma:software} dan Maurice Marx~\cite{marx:software}.

\section{Rumusan Masalah}
\label{sec:rumusan}

Berdasarkan uraian pada bagian sebelumnya, berikut merupakan masalah-masalah yang hendak diselesaikan oleh penelitian ini:
\iffalse

\lionov{masalah pertama itu ukuran apa saja yang bisa digunakan, kedua bagaimana membuat model pergerakan kolektif yang memanfaatkan ukuran lintasan, yang ketiga bagaimana membuat algoritmanya. Soalnya yang poin 2 dan 3 kan gak dilakukan, elu gak ngebahas berbagai definisi formal, ukuran, teknik, dan algoritma (karena pertanyaannya ``apa aja''}

\fi
\begin{enumerate}
    \item Seperti apakah definisi pergerakan kolektif yang dapat mengatasi masalah identifikasi pada data pergerakan yang memiliki kasus perbedaan arah dan kecepatan? 
    \item Bagaimana langkah-langkah identifikasi pergerakan kolektif berdasarkan definisi pergerakan kolektif yang dibuat?  
    \item Bagaimana cara mengevaluasi efektivitas definisi pergerakan kolektif dalam mengidentifikasi pergerakan kolektif pada data pergerakan?
\end{enumerate}

\section{Tujuan}
\label{sec:tujuan}  

Berdasarkan masalah-masalah yang telah dirumuskan pada bagian sebelumnya, maka penelitian ini bertujuan untuk:

\begin{enumerate}
    \item Membuat definisi pergerakan kolektif baru yang mampu mengatasi masalah identifikasi pada kasus perbedaan arah dan kecepatan.
    \item Membuat dan mengimplementasikan algoritma yang dapat mengidentifikasi pergerakan kolektif berdasarkan definisi formal yang telah dibuat.
    \item Melakukan evaluasi terhadap efektivitas definisi formal pergerakan kolektif yang sudah dibuat secara kuantitatif dan kualitatif.
\end{enumerate}

\section{Batasan Masalah}
\label{sec:batasan}

Dalam pengerjaannya, skripsi ini hanya akan membahas masalah-masalah berikut:

\begin{enumerate}
    \item Data pergerakan yang akan dianalisis bersumber dari pergerakan entitas dalam ruang \textit{euclidean} dua dimensi $\mathbb{R}^2$.
    \item Perangkat lunak yang digunakan untuk melakukan visualisasi hasil identifikasi rombongan merupakan perangkat lunak yang sudah tersedia dan bukan merupakan bagian dari implementasi perangkat lunak dalam penelitian ini.
    \item Evaluasi hasil eksperimen tidak akan memperhitungkan kompleksitas algoritma yang digunakan, melainkan hanya akan mengukur relevansi hasil identifikasi pergerakan kolektif yang dihasilkan oleh algoritma identifikasi dibandingkan dengan hasil identifikasi pergerakan kolektif yang dilakukan oleh manusia.
\end{enumerate}

\section{Metodologi}
\label{sec:metlit}

Metodologi penelitian yang akan digunakan dalam skripsi ini adalah sebagai berikut:

\begin{enumerate}
    \item Melakukan studi pustaka mengenai berbagai ukuran dan atribut yang terdapat pada sebuah lintasan dari entitas yang bergerak.
    \item Melakukan studi pustaka mengenai berbagai definisi formal pergerakan kolektif yang sudah dibuat sebelumnya.
    \item Membuat upaya penyelesaian masalah identifikasi pergerakan kolektif pada kasus perbedaan arah dan kecepatan menggunakan ukuran-ukuran yang terdapat dalam sebuah lintasan.
    \item Membuat definisi pergerakan kolektif baru yang mampu mengatasi masalah identifikasi pergerakan kolektif pada kasus perbedaan arah dan kecepatan sesuai dengan upaya penyelesaian masalah yang dibuat.
    \item Membuat algoritma yang mampu mengidentifikasi pergerakan kolektif dari sebuah pergerakan kolektif yang sesuai dengan definisi baru yang dibuat.
    \item Mengimplementasikan algoritma identifikasi pergerakan kolektif yang dibuat menjadi sebuah perangkat lunak.
    \item Melakukan eksperimen dengan mengidentifikasi pergerakan kolektif yang sesuai dengan definisi pergerakan kolektif baru yang dibuat menggunakan perangkat lunak identifikasi pada data pergerakan dunia nyata.
    \item Melakukan visualisasi hasil identifikasi pergerakan kolektif menggunakan model pewarnaan dalam video rekaman data pejalan kaki dunia nyata menggunakan perangkat lunak visualisasi.
    \item Melakukan evaluasi terhadap hasil identifikasi pergerakan kolektif secara kuantitatif pada hasil eksperimen dengan menghitung nilai \textit{precision}, \textit{recall}, dan \textit{F1 score} dengan membandingkan hasil identifikasi yang dihasilkan perangkat lunak dan hasil identifikasi yang dibuat oleh manusia pada beberapa data pergerakan.
    \item Melakukan evaluasi terhadap hasil identifikasi pergerakan kolektif secara kualitatif dengan menganalisis hasil visualisasi dari hasil identifikasi yang dihasilkan perangkat lunak pada rekaman video data pergerakan pejalan kaki dunia nyata.
    \item Menuliskan proses, hasil eksperimen, serta hasil analisis yang diperoleh dari penelitian.
\end{enumerate}

\section{Sistematika Pembahasan}
\label{sec:sispem}

Penelitian ini akan dibahas dalam tujuh bab yang masing-masing akan berisi:

\begin{enumerate}
    \item \textbf{Bab 1 Pendahuluan}
    
    Bab 1 akan berisi pembahasan mengenai latar belakang tentang pergerakan, lintasan, pergerakan kolektif, serta masalah-masalah yang terdapat pada definisi-definisi pergerakan kolektif sebelumnya. Bab ini juga membahas rumusan masalah, tujuan yang hendak dicapai, batasan masalah, metodologi penelititan, serta sistematika penulisan dari skripsi ini.
    
    \item \textbf{Bab 2 Dasar Teori}
    
    Bab 2 akan berisi dasar teori mengenai lintasan, sumber data lintasan, ukuran yang terdapat pada lintasan, penghitungan kemiripan lintasan, serta pergerakan kolektif secara umum. Pada bab ini juga akan dibahas secara singkat mengenai definisi pergerakan kolektif yang sudah dibuat sebelumnya.
    
    \item \textbf{Bab 3 Analisis}
    
    Bab 3 akan berisi pembahasan mengenai masalah-masalah yang terdapat pada definisi-definisi formal pergerakan kolektif yang sudah dibuat sebelumnya serta analisis mengenai cara yang diusulkan untuk mengatasi masalah-masalah tersebut.
    
    Selain itu, bab ini juga akan membahas mengenai definisi formal pergerakan kolektif baru yang dibuat dan algoritma yang dapat digunakan untuk mengidentifikasi pergerakan kolektif berdasarkan definisi tersebut.
    
    \item \textbf{Bab 4 Perancangan}
    
    Bab 4 akan berisi pembahasan mengenai rancangan perangkat lunak identifikasi rombongan yang akan dibuat. Bab ini juga akan membahas mengenai modul-modul perangkat lunak yang akan dibuat serta struktur-struktur data yang akan digunakan selama proses identifikasi rombongan dilakukan.
    
    \item \textbf{Bab 5 Implementasi}
    
    Bab 5 akan berisi pembahasan mengenai detil implementasi rancangan perangkat lunak identifikasi rombongan yang sudah dibuat pada bab sebelumnya. Dalam bab ini, akan dibahas mengenai bahasa pemrograman, paradigma, serta pustaka-pustaka yang digunakan dalam mengimplementasikan rancangan perangkat lunak yang telah dibuat.
    
    \item \textbf{Bab 6 Eksperimen}
    
    Bab 6 akan berisi eksperimen yang dilakukan menggunakan perangkat lunak identifikasi rombongan yang sudah dibuat pada bab sebelumnya. Dalam bab ini, akan dijelaskan mengenai metodologi dari proses evaluasi yang digunakan, sumber data pergerakan yang digunakan, serta penyajian dan pembahasan hasil eksperimen yang dilakukan terhadap data pergerakan menggunakan perangkat lunak identifikasi rombongan.
    
    \item \textbf{Bab 7 Kesimpulan}
    
    Bab 7 akan berisi kesimpulan dari proses penelitian yang sudah dilakukan. Dalam bab ini, terdapat pembahasan mengenai pencapaian tujuan penelitian yang telah dinyatakan pada Bab 1 serta jawaban mengenai rumusan masalah penelitian. Bab ini juga akan membahas mengenai saran penelitian lanjutan yang relevan dengan topik yang sama.
\end{enumerate}
