%versi 3 (18-12-2016)
\chapter{Kode Program}
\label{lamp:A}

%terdapat 2 cara untuk memasukkan kode program
% 1. menggunakan perintah \lstinputlisting (kode program ditempatkan di folder yang sama dengan file ini)
% 2. menggunakan environment lstlisting (kode program dituliskan di dalam file ini)
% Perhatikan contoh yang diberikan!!
%
% untuk keduanya, ada parameter yang harus diisi:
% - language: bahasa dari kode program (pilihan: Java, C, C++, PHP, Matlab, C#, HTML, R, Python, SQL, dll)
% - caption: nama file dari kode program yang akan ditampilkan di dokumen akhir
%
% Perhatian: Abaikan warning tentang textasteriskcentered!!
%

\begin{lstlisting}[language=C++, caption=Implementasi modul \texttt{rombongan.cpp}, label={lamp:module-rombongan}]
#include "rombongan.h"
#include "spatial.h"
#include "entity.h"
#include "utils.h"
#include "io.h"
#include "parser.h"
#include <cmath>
#include <vector>
#include <set>
#include <unordered_map>
#include <algorithm>
#include <iostream>
#include <map>

typedef std::unordered_map<unsigned int, std::vector<std::vector<double> > > trajectory_map;
typedef std::unordered_map<unsigned int, std::vector<double> > direction_map;
typedef std::map<std::vector<unsigned int>, std::vector<std::pair<unsigned int, unsigned int> > > rombongan_lifespan;

/**
 * Determine is a list is a imperfect sublist of another.
 *
 * @param a - first list
 * @param b - second list
 * @return `true` if a list is a imperfect sublist of another,
 * `false` otherwise
 */
bool is_sublist(
    const std::vector<unsigned int>& a,
    const std::vector<unsigned int>& b
) {
    std::set<unsigned int> temp_container;
    temp_container.insert(a.begin(), a.end());
    temp_container.insert(b.begin(), b.end());

    return a.size() != b.size() &&
        (
            temp_container.size() == a.size() ||
            temp_container.size() == b.size()
        );
}

/**
 * Determine if an entity is present on a time interval
 * 
 * @param target_entity - target entity
 * @param interval - time interval
 * 
 * @return `true` if target is present in all timestamps,
 * `false` otherwise.
 */
bool on_interval(
    const Entity& target_entity,
    const std::pair<unsigned int, unsigned int>& interval
) {
    std::vector<std::vector<double> > trajectory = {
        target_entity.trajectories.begin() + interval.first,
        target_entity.trajectories.begin() + interval.second
    };

    for (size_t t_itr = 0; t_itr < trajectory.size(); t_itr++) {
        if (std::isnan(trajectory[t_itr][0])) {
            return false;
        }
    }

    return true;
}

/**
 * Merge similar duration from child to parent, a.k.a deleting it
 * 
 * @param parent parent duration
 * @param child child duration
 */
void deduplicate_duration(
    std::vector<std::pair<unsigned int, unsigned int> >& parent,
    std::vector<std::pair<unsigned int, unsigned int> >& child
) {
    std::vector<std::pair<unsigned int, unsigned int> > deleted;

    for (size_t parent_itr = 0; parent_itr < parent.size(); parent_itr++) {
        std::pair<unsigned int, unsigned int> parent_duration = parent[parent_itr];

        for (size_t child_itr = 0; child_itr < child.size(); child_itr++) {
            std::pair<unsigned int, unsigned int> child_duration = child[child_itr];

            if (
                (parent_duration == child_duration) ||
                (child_duration.first > parent_duration.first && child_duration.second < parent_duration.second)
            ) {
                deleted.push_back(child_duration);
            }
        }
    }

    for (size_t del_itr = 0; del_itr < deleted.size() && child.size() > 0; del_itr++) {
        auto position = std::find(
            child.begin(),
            child.end(),
            deleted[del_itr]
        );

        if (position != child.end()) {
            child.erase(position);
        }
    }
}

/**
 * Clean rombongan identification result by merging
 * sub-rombongan into parent rombongan.
 * 
 * @param raw_result raw idenfication result
 *
 * @return cleaned identification result
 */
std::vector<Rombongan> deduplicate(
    std::vector<Rombongan>& raw_result
) {
    if (raw_result.size() == 0) {
        return raw_result;
    }

    std::sort(raw_result.begin(), raw_result.end());

    for (size_t curr_itr = 0; curr_itr < raw_result.size() - 1; curr_itr++) {
        if (raw_result[curr_itr].duration.size() == 0) {
            continue;
        }

        for (size_t other_itr = curr_itr + 1; other_itr < raw_result.size(); other_itr++) {
            if (raw_result[other_itr].duration.size() == 0) {
                continue;
            }

            if (
                is_sublist(raw_result[curr_itr].members, raw_result[other_itr].members)
            ) {
                deduplicate_duration(
                    raw_result[curr_itr].duration,
                    raw_result[other_itr].duration
                );

                
            }
        }
    }

    std::vector<Rombongan> clean_result;

    for (auto member: raw_result) {
        if (member.duration.size() > 0) {
            clean_result.push_back(member);
        }
    }

    sort(
        clean_result.begin(),
        clean_result.end(),
        [](const Rombongan& a, const Rombongan& b) -> bool {
            return a.duration[0].first < b.duration[0].first;
        }
    );

    return clean_result;
}


/**
 * Get sub trajectories from all entities for a time interval
 * 
 * @param entities input entities
 * @param interval time interval
 */
trajectory_map get_sub_trajectories(
    const std::vector<Entity>& entities,
    const std::pair<unsigned int, unsigned int>& interval
) {
    trajectory_map sub_trajectories;

    for (size_t itr = 0; itr < entities.size(); itr++) {
        Entity curr = entities[itr];

        sub_trajectories[curr.id] = {
            curr.trajectories.begin() + interval.first,
            curr.trajectories.begin() + interval.second
        };
    }

    return sub_trajectories;
}

/**
 * Get directional vectors for all entities in a time interval
 * 
 * @param entities list of entities
 * @param interval time interval
 */
direction_map get_directional_vectors(
    const std::vector<Entity>& entities,
    const std::pair<unsigned int, unsigned int>& interval
) {
    unsigned int dimension = entities[0].trajectories[0].size();

    direction_map directional_vectors;

    for (size_t itr = 0; itr < entities.size(); itr++) {
        Entity curr = entities[itr];
        std::vector<double> end_pos = curr.trajectories[interval.second - 1];
        std::vector<double> start_pos = curr.trajectories[interval.first];

        for (size_t curr_dimension = 0; curr_dimension < dimension; curr_dimension++) {
            directional_vectors[curr.id].push_back(
                end_pos[curr_dimension] - start_pos[curr_dimension]
            );
        }
    }

    return directional_vectors;
}

/**
 * Extends a rombongan duration
 * 
 * @param groups list of possible rombongan
 * @param other tested entity
 * @param sub_trajectories List of sub trajectories for the current time interval
 * @param direction_vector List of directional vectors for the current time interval
 * @param params Identification parameters
 */
void extend_current_rombongan(
    std::vector<std::vector<unsigned int> >& groups,
    const Entity& other,
    trajectory_map& sub_trajectories,
    direction_map& direction_vector,
    const Parameter& params
) {
    double r = params.r;
    double cs = params.cs;
    unsigned int m = params.m;

    for (size_t groups_itr = 0; groups_itr < groups.size(); groups_itr++) {
        unsigned int similarity_count = 0;
        unsigned int limit = params.c == 0 ? groups[groups_itr].size() : params.c;

        for (size_t member_itr = 0; member_itr < groups[groups_itr].size() && similarity_count < limit; member_itr++) {
            unsigned int member_id = groups[groups_itr][member_itr];

            if (member_id == other.id) {
                continue;
            }
        
            double dtw_distance = calculate_dtw_distance(
                sub_trajectories[other.id],
                sub_trajectories[member_id]
            );

            double cosine_similarity = calculate_cosine_similarity(
                direction_vector[other.id],
                direction_vector[member_id]
            );

            // make sure that the distance is not zero to avoid
            // similarity checking when two entities
            // doesn't appear in the current timeframe.
            similarity_count += !std::isnan(dtw_distance) &&
                dtw_distance <= r &&
                cosine_similarity >= cs;
        }

        if (similarity_count >= limit) {
            groups[groups_itr].push_back(other.id);
        }
    }
}

/**
 * Extend identified rombongan duration by either appending
 * or replacing duration pairs
 * 
 * @param groups - list of identified rombongan
 * @param current_groups - list of newly identified rombongan
 * @param interval - time interval
 */
void extend_rombongan_duration(
    rombongan_lifespan& groups,
    std::vector<std::vector<unsigned int> >& current_groups,
    const std::pair<unsigned int, unsigned int>& interval
) {
    unsigned int start = interval.first;
    unsigned int end = interval.second;

    for (std::vector<unsigned int> group: current_groups) {
        std::vector<std::pair<unsigned int, unsigned int> > time_list = groups[group];

        if (
            time_list.size() == 0 ||
            start == 0 ||
            (end - 1) != time_list[time_list.size() - 1].second
        ) {
            groups[group].push_back({ start, end });
        } else {
            groups[group][time_list.size() - 1] = { 
                groups[group][time_list.size() - 1].first,
                end
            };
        }
    }
}

/**
 * Determine if a rombongan candidate is not a subrombongan
 * of another
 * 
 * @param candidate - candidate rombongan
 * @param groups - list of identified rombongan
 * 
 * @return `true` if candidate is not a subrombongan,
 * `false` otherwise
 */
bool is_new_rombongan(
    const std::vector<unsigned int>& candidate,
    const std::vector<std::vector<unsigned int> >& groups
) {
    for (size_t group_itr = 0; group_itr < groups.size(); group_itr++) {
        if (
            groups[group_itr].size() > candidate.size() &&
            is_sublist(groups[group_itr], candidate)
        ) {
            return false;
        }
    }

    return true;
}

/**
 * Identify rombongan from a set of moving entities.
 * 
 * @param entities - list of moving entities in two-dimensional space
 */
std::vector<Rombongan> identify_rombongan(
    const std::vector<Entity>& entities,
    const Parameter& params
) {
    unsigned int m = params.m;
    unsigned int k = params.k;
    double r = params.r;
    double cs = params.cs;

    rombongan_lifespan rombongan_list;
    
    unsigned int frame_count = entities[0].trajectories.size();

    for (size_t end = k; end < frame_count; end++) {
        unsigned int start = end - k;

        std::vector<std::vector<unsigned int> > current_rombongan;

        trajectory_map sub_trajectories = get_sub_trajectories(
            entities,
            { start, end }
        );
        direction_map direction_vector = get_directional_vectors(
            entities,
            { start, end }
        );

        for (size_t curr_itr = 0; curr_itr < entities.size() - 1; curr_itr++) {
            Entity curr = entities[curr_itr];

            if (!on_interval(curr, { start, end })) {
                continue;
            }

            std::vector<std::vector<unsigned int> > group_ids;

            for (size_t other_itr = curr_itr + 1; other_itr < entities.size(); other_itr++) {
                Entity other = entities[other_itr];

                if (!on_interval(other, { start, end })) {
                    continue;
                }

                extend_current_rombongan(
                    group_ids,
                    other,
                    sub_trajectories,
                    direction_vector,
                    params
                );

                // can entity `curr` create a subgroup with `other`?
                double dtw_distance = calculate_dtw_distance(
                    sub_trajectories[other.id],
                    sub_trajectories[curr.id]
                );

                double cosine_similarity = calculate_cosine_similarity(
                    direction_vector[other.id],
                    direction_vector[curr.id]
                );

                if (
                    !std::isnan(dtw_distance) &&
                    dtw_distance <= r &&
                    cosine_similarity >= cs &&
                    is_new_rombongan({ curr.id, other.id }, group_ids)
                ) {
                    group_ids.push_back({ curr.id, other.id });
                }
            }

            for (size_t itr_group = 0; itr_group < group_ids.size(); itr_group++) {
                if (group_ids[itr_group].size() >= m) {
                    current_rombongan.push_back(group_ids[itr_group]);
                }
            }
        }

        if (current_rombongan.size() > 0) {
            extend_rombongan_duration(
                rombongan_list,
                current_rombongan,
                { start, end }
            );
        }

        std::cout << "Finished processing range ";
        std::cout << "[" << start << ", " << end << "]" << std::endl;
    }

    std::vector<Rombongan> raw_result;

    for (auto const& [group, duration]: rombongan_list) {
        raw_result.push_back({
            group,
            duration
        });
    }

    return params.redundant ? raw_result : deduplicate(raw_result);
}
\end{lstlisting}

\begin{lstlisting}[language=C++, caption=Implementasi 
modul \texttt{spatial.cpp},label={lamp:module-spatial}]
#include <vector>
#include <limits>
#include <cmath>
#include <algorithm>
#include <stdexcept>

/**
 * Calculate the euclidean distance between two points in
 * n-dimensional space.
 * 
 * @param a the first point
 * @param b the second point
 * @return Euclidean distance between two points `a` and `b`
 */
double calculate_euclidean_distance(
    const std::vector<double> &a,
    const std::vector<double> &b
) {
    if (a.size() != b.size()) {
        throw std::invalid_argument(
            "Both trajectories must reside in the same dimensional space."
        );
    }

    double distance = 0.0;

    for (size_t it = 0; it < a.size(); it++) {
        // return immediately when one of the position is not recorded
        if (std::isnan(a[it]) || std::isnan(b[it])) {
            return std::nan("");
        }

        distance += pow(a[it] - b[it], 2);
    }

    return sqrt(distance);
}

/**
 * Calculate dynamic time warping distance between two distinct
 * trajectories.
 * 
 * @param a the first trajectory
 * @param b the second trajectory
 * @return Dynamic time warping distance between two trajectories
 * `a` and `b`
 */
double calculate_dtw_distance(
    const std::vector<std::vector<double> >& a,
    const std::vector<std::vector<double> >& b
) {
    if (a.size() == 0 || b.size() == 0) {
        throw std::invalid_argument("Both trajectories must not be empty.");
    }

    // create a 2-dimensional array to store the DTW distances
    const size_t x_lim = a.size() + 1;
    const size_t y_lim = b.size() + 1;
    double dtw[x_lim][y_lim];

    // initialize all illegal first pairings with infinity
    for (size_t i = 0; i < x_lim; i++) {
        dtw[i][0] = std::numeric_limits<double>::max();        
    }

    for (size_t i = 0; i < y_lim; i++) {
        dtw[0][i] = std::numeric_limits<double>::max();
    }

    // Pair the first point of `a` with the first point of `b`
    dtw[0][0] = 0;

    // Calculate DTW distance
    for (size_t i = 1; i < x_lim; i++) {
        for (size_t j = 1; j < y_lim; j++) {
            double cost = calculate_euclidean_distance(
                a[i - 1],
                b[j - 1]
            );

            // return immediately when trajectory isn't recorded
            if (std::isnan(cost)) {
                return std::nan("");
            }

            double min = std::min(
                dtw[i - 1][j],
                std::min(
                    dtw[i][j - 1],
                    dtw[i - 1][j - 1]
                )
            );

            dtw[i][j] = cost + min;
        }
    }

    return dtw[a.size()][b.size()];
}

/**
 * Calculate cosine similarity between two n-dimensional vector.
 * 
 * @param a - the first vector
 * @param b - the second vector
 * @return 
 */
double calculate_cosine_similarity(
    const std::vector<double>& a,
    const std::vector<double>& b
) {
    // sanity check
    if (a.size() != b.size()) {
        throw std::invalid_argument(
            "Both vector must reside in the same dimensional space."
        );
    }

    double len_a = 0.0;
    double len_b = 0.0;
    double dot_product = 0.0;

    for (size_t itr = 0; itr < a.size(); itr++) {
        // return immediately when one of the position is not recorded
        if (std::isnan(a[itr]) || std::isnan(b[itr])) {
            return std::nan("");
        }

        dot_product += a[itr] * b[itr];

        len_a += pow(a[itr], 2);
        len_b += pow(b[itr], 2);
    }

    len_a = sqrt(len_a);
    len_b = sqrt(len_b);

    return dot_product / (len_a * len_b);
}
\end{lstlisting}

\begin{lstlisting}[language=C++, caption=Implementasi 
modul \texttt{parser.cpp},label={lamp:module-parser}]
#include "entity.h"
#include "io.h"
#include "parser.h"
#include <iostream>
#include <sstream>
#include <fstream>
#include <string>
#include <vector>
#include <algorithm>
#include <cmath>
#include <unordered_set>
#include <unordered_map>
#include <filesystem>
#include <map>
#include <stdexcept>

// PI hack
const double PI = std::acos(-1);

typedef std::map<double, std::unordered_map<unsigned int, std::vector<double> > > frame_to_entity;
typedef std::unordered_map<unsigned int, std::vector<std::vector<double> > > entity_to_frame;

/**
 * Parse data from a text file and return it as list of moving
 * entities
 * 
 * @param name Name of the file to be read from
 * @param path File path to read from
 * @return List of moving entities with their trajectories
 */
MovementData parse_data(
    const std::string& name,
    const std::string& path
) {
    std::ifstream file;
    std::string line;

    double frame_time, id, x_pos, y_pos;

    auto filepath = (std::filesystem::path(path) / "input" / name)
        .make_preferred()
        .replace_extension(".txt");

    std::cout << "Attempting to read source data from: ";
    std::cout << filepath.string() << std::endl; 

    file.open(filepath);

    if (!file) {
        throw std::invalid_argument("Source data doesn't exist!");
    }

    std::unordered_set<unsigned int> id_frame;
    std::unordered_set<unsigned int> id_list;

    frame_to_entity data_map;
    entity_to_frame trajectory_map;

    while (std::getline(file, line)) {
        std::istringstream line_stream(line);

        line_stream >> frame_time >> id >> x_pos >> y_pos >> y_pos;

        id_list.insert(id);

        data_map[frame_time][id] = { x_pos, y_pos }; 
    }

    file.close();

    std::vector<double> frames;

    for (auto const& [frame, position]: data_map) {
        frames.push_back(frame);
    
        for (auto const& [id, plane]: position) {
            trajectory_map[id].push_back(
                plane  
            );

            id_frame.insert(id);
        }

        for (auto const id: id_list) {
            if (id_frame.find(id) == id_frame.end()) {
                trajectory_map[id].push_back({ std::nan(""), std::nan("") });
            }
        }

        id_frame.clear();
    }

    std::vector<Entity> entities;

    for (auto const& [id, frame]: trajectory_map) {
        entities.push_back(
            { id, frame }
        );
    }

    // make sure that entities are sorted to prevent random pairings
    sort(entities.begin(), entities.end());

    return {
        entities,
        frames
    };
}

/**
 * Parse rombongan identification parameters from input
 * arguments
 * 
 * @param args input arguments
 * @return rombongan identification parameters
 */
Parameter parse_arguments(
    const Arguments& args
) {
    Parameter params;

    params.m = args.entities;
    params.c = args.closeness;
    params.k = std::ceil(args.interval);
    params.r = args.range * args.interval;
    params.cs = cos(args.angle * PI / 180);
    params.redundant = args.redundant;

    return params;
}

/**
 * Parse expected result from a text file and return it
 * as an array of entity id
 * 
 * @param name expected result filename
 * @param path expected result filepath
 * 
 * @return list of groups, which is a list of id
 */
std::vector<std::vector<unsigned int> > parse_expected_result(
    const std::string& name,
    const std::string& path
) {
    std::ifstream file;
    std::string line;
    unsigned int bucket;

    double frame_time, id, x_pos, y_pos;

    auto filepath = (std::filesystem::path(path) / "result" / name)
        .make_preferred()
        .replace_extension(".txt");

    std::cout << "Attempting to read expected results from: ";
    std::cout << filepath.string() << std::endl; 

    file.open(filepath);

    if (!file) {
        throw std::invalid_argument("Expected result doesn't exist!");
    }

    std::vector<std::vector<unsigned int> > expected_result;

    while (std::getline(file, line)) {
        std::vector<unsigned int> group;

        std::istringstream line_stream(line);

        while (line_stream >> bucket) {
            group.push_back(bucket);
        }

        // sort the ids to make sure that default
        // comparator function work.
        sort(group.begin(), group.end());

        expected_result.push_back(group);
    }

    return expected_result;
}
\end{lstlisting}

\begin{lstlisting}[language=C++, caption=Implementasi 
modul \texttt{io.cpp},label={lamp:module-io}]
#include "io.h"
#include "eval.h"
#include "utils.h"
#include <iostream>
#include <iomanip>
#include <fstream>
#include <iterator>
#include <filesystem>
#include <stdexcept>
#include <argparse/argparse.hpp>

/**
 * Parse user inputs from command-line arguments.
 * 
 * @param argc arguments count
 * @param argv actual arguments
 * 
 * @return list of parsed arguments.
 */
Arguments read_arguments(int argc, char *argv[]) {
    argparse::ArgumentParser program("rombongan");

    program.add_argument("data")
        .help("Nama berkas text dari sumber data yang ingin digunakan")
        .required();

    program.add_argument("entities")
        .help("Jumlah entitas minimum anggota rombongan")
        .required()
        .action([](const std::string &value) { return std::stoi(value); });
    
    program.add_argument("interval")
        .help("Interval waktu minimum pergerakan bersama secara konsekutif dalam satuan frame")
        .required()
        .action([](const std::string &value) { return std::stod(value); });

    program.add_argument("range")
        .help("Jarak entitas maksimum antar anggota rombongan dalam satu waktu")
        .required()
        .action([](const std::string &value) { return std::stod(value); });

    program.add_argument("angle")
        .help("Beda sudut maksimum antar entitas dalam rombongan dalam derajat")
        .required()
        .action([](const std::string &value) { return std::stod(value); });
    
    program.add_argument("closeness")
        .help("Jumlah entitas minimum yang dekat agar dapat tergabung dalam rombongan")
        .required()
        .action([](const std::string &value) { return std::stoi(value); });

    program.add_argument("-p", "--path")
        .help("Direktori sumber data lintasan. Relatif terhadap direktori saat ini.")
        .default_value(
            (std::filesystem::current_path() / "data")
                .make_preferred()
                .string()
        )
        .action([](const std::string &value) {
            return (std::filesystem::current_path() / value)
                .make_preferred()
                .string();
        });

    program.add_argument("-r", "--redundant")
        .help("Menentukan apakah rombongan redundan disimpan atau tidak")
        .default_value(false)
        .implicit_value(true);

    try {
        program.parse_args(argc, argv);
    } catch (const std::runtime_error &err) {
        std::cout << err.what() << "\n";
        std::cout << program;
        exit(0);
    }

    auto data = program.get<std::string>("data");
    auto entities = program.get<int>("entities");

    if (entities < 2) {
        throw std::invalid_argument(
            "Jumlah entitas minimum harus lebih besar dari 1."
        );
    }

    auto interval = program.get<double>("interval");

    if (interval < 0) {
        throw std::invalid_argument(
            "Durasi pergerakan bersama harus merupakan bilangan positif 1."
        );
    }

    auto range = program.get<double>("range");

    if (range <= 0.0f) {
        throw std::invalid_argument(
            "Jarak entitas maksimum harus merupakan bilangan positif."
        );
    }

    auto angle = program.get<double>("angle");

    auto closeness = program.get<int>("closeness");

    if (closeness < 0) {
        throw std::invalid_argument(
            "Jumlah entitas minimum yang dekat harus merupakan bilangan tak negatif"
        );
    }

    auto path = program.get<std::string>("-p");
    auto redundancy = program.get<bool>("-r");

    Arguments p;

    p.source = data;
    p.entities = entities;
    p.closeness = closeness;
    p.interval = interval;
    p.range = range;
    p.angle = angle;
    p.path = path;
    p.redundant = redundancy;

    return p;
}

/**
 * Write rombongan identification result into a text file.
 * 
 * @param result rombongan identification result
 * @param frames frame list
 * @param params arguments passed to the program
 * @param score predicition score
 */
void write_result(
    const std::vector<Rombongan>& result,
    const std::vector<double>& frames,
    const Arguments& args,
    const Score& score
) {
    std::ofstream file_stream;

    std::filesystem::path dir_path = 
        (std::filesystem::current_path() / "data" / "output")
        .make_preferred();

    if (!std::filesystem::exists(dir_path)) {
        std::filesystem::create_directory(dir_path);
        std::filesystem::permissions(
            dir_path,
            std::filesystem::perms::owner_all |
            std::filesystem::perms::group_all |
            std::filesystem::perms::others_read,
            std::filesystem::perm_options::add
        );
    }

    auto now = get_current_time();
    auto output_filename = args.source + "-" + std::to_string(now);

    auto [precision, recall, f1_score] = score;

    auto output_filepath = (dir_path / output_filename)
        .make_preferred()
        .replace_extension(".txt");
    
    file_stream.open(
        output_filepath,
        std::ofstream::out | std::ofstream::trunc
    );

    if (file_stream.is_open()) {
        // write the parameters in csv-like format
        file_stream << "data, m, k, r, theta, n" << std::endl;
        file_stream << args.source << ", ";
        file_stream << args.entities << ", ";
        file_stream << args.interval << ", ";
        file_stream << args.range << ", ";
        file_stream << args.angle << ", ";
        file_stream << args.closeness;
        file_stream << std::endl << std::endl;

        std::streamsize ss = std::cout.precision();

        file_stream << "precision, recall, f1_score" << std::endl;
        file_stream << std::setprecision(3);
        file_stream << precision << ", ";
        file_stream << recall << ", ";
        file_stream << f1_score;
        file_stream << std::endl << std::endl;

        file_stream << std::setprecision(ss);

        // initialize csv-like header for identification result
        file_stream << "members, duration" << std::endl;

        for (Rombongan group: result) {
            std::vector<unsigned int> members = group.members;
            std::vector<std::pair<unsigned int, unsigned int> > duration = group.duration;

            for (size_t i = 0; i < members.size(); i++) {
                if (i > 0) {
                    file_stream << " ";
                }

                file_stream << members[i];
            }

            file_stream << ", ";

            for (size_t i = 0; i < duration.size(); i++) {
                if (i > 0) {
                    file_stream << " ";
                }

                double start = frames[duration[i].first];
                double end = frames[duration[i].second];

                file_stream << start << "-" << end;
            }

            file_stream << std::endl;
        }

        file_stream.close();

        std::cout << "Successfully written identification result to ";
        std::cout << output_filename << ".txt" << std::endl;
    } else {
        throw std::runtime_error(
            "Failed to open output stream due to lack of permissions."
        );
    }
}
\end{lstlisting}


\begin{lstlisting}[language=C++, caption=Implementasi 
modul \texttt{eval.cpp},label={lamp:module-eval}]
#include "eval.h"
#include "rombongan.h"
#include <vector>
#include <cmath>
#include <algorithm>

/**
 * Find a group in a list of groups
 * 
 * @param source source list
 * @param value target value
 * 
 * @return `true` if value exists in source, `false`
 * otherwise
 */
bool find(
    const std::vector<std::vector<unsigned int> >& source,
    const std::vector<unsigned int>& value
) {
    return std::find(source.begin(), source.end(), value) != source.end();
}

/**
 * Calculate precision for the prediciton based on a ground truth
 * 
 * 
 * @param expected_result - ground truth
 * @param result - predicition
 * 
 * @return precision value.
 */
double calculate_precision(
    const std::vector<std::vector<unsigned int> >& expected_result,
    const std::vector<std::vector<unsigned int> >& result
) {
    double tp = 0;
    double fp = 0;

    for (size_t itr = 0; itr < result.size(); itr++) {
        find(expected_result, result[itr]) ?
            tp++ :
            fp++;
    }

    double precision = tp / (tp + fp);

    return std::isnan(precision) ? 0 : precision;
}

/**
 * Calculate recall for the prediciton based on a ground truth
 * 
 * @param expected_result - ground truth
 * @param result - predicition
 * 
 * @return recall value.
 */
double calculate_recall(
    const std::vector<std::vector<unsigned int> >& expected_result,
    const std::vector<std::vector<unsigned int> >& result
) {
    double tp = 0;
    double fn = 0;

    for (size_t itr = 0; itr < expected_result.size(); itr++) {
        find(result, expected_result[itr]) ?
            tp++ :
            fn++;
    }

    double recall = tp / (tp + fn);

    return std::isnan(recall) ? 0 : recall;
}

/**
 * Calculate f1 score for the prediciton based on a ground truth
 * 
 * @param expected_result - ground truth
 * @param result - predicition
 * 
 * @return evaluation result, including precision, recall, and F1 score
 */
Score calculate_score(
    const std::vector<std::vector<unsigned int> >& expected_result,
    const std::vector<Rombongan>& result
) {
    std::vector<std::vector<unsigned int> > result_ids;

    for (size_t itr = 0; itr < result.size(); itr++) {
        result_ids.push_back(result[itr].members);
    }

    double precision = calculate_precision(expected_result, result_ids);
    double recall = calculate_recall(expected_result, result_ids);
    double f1_score = 2 * (precision * recall) / (precision + recall);

    if (std::isnan(f1_score)) {
        f1_score = 0;
    }

    return {
        precision,
        recall,
        f1_score
    };
}
\end{lstlisting}