\chapter{Implementasi}
\label{chap:implementasi}

Setelah rancangan perangkat lunak berhasil dibuat, tahap selanjutnya yang akan dikerjakan adalah tahap implementasi perangkat lunak. Dalam tahap implementasi perangkat lunak, rancangan perangkat lunak yang sudah dibuat akan diubah menjadi kode program yang sesuai dengan rancangan dan tujuan perangkat lunak.

Pada skripsi ini, perangkat lunak identifikasi rombongan akan dibuat menggunakan paradigma paradigma pemrograman fungsional. C++ dipilih sebagai bahasa pemrograman yang digunakan untuk mengimplementasikan perangkat lunak. Pemilihan C++ sebagai bahasa pemrograman didasarkan pada performa dari bahasa pemrograman yang baik dan adanya seperangkat perkakas pemrograman berupa STL yang mempermudah proses pembuatan perangkat lunak.

\section{Format Data Rombongan}
\label{sec:input-structure}

Data pergerakan pejalan kaki akan disimpan dalam sebuah berkas teks dengan format \texttt{.txt} dengan \textit{encoding} UTF-8. Berkas tersebut menyimpan informasi dari seluruh entitas yang tercatat dalam data pergerakan pejalan kaki. Berkas tersebut merupakan hasil ekstraksi dari informasi yang terdapat dalam sebuah video rekaman pergerakan pejalan kaki.

Informasi yang terdapat dalam data pejalan kaki disimpan dalam bentuk \textit{frame} yang telah diekstrak dari video rekaman pergerakan pejalan kaki. Sebuah \textit{frame} mengandung informasi dalam bentuk \texttt{<nomor-frame> <nomor-identitas> <posisi-x> <posisi-z> <posisi-y> <kecepatan-x> <kecepatan-z> <kecepatan-y>} yang ditulis dalam satu baris pada berkas teks.

Nilai dari \texttt{<nomor-frame>} menunjukkan \textit{frame} pada video rekaman yang memiliki informasi yang terdapat pada baris teks. Nilai dari \texttt{<nomor-identitas>} menunjukkan nomor identitas unik dari entitas. Nilai dari \texttt{<posisi-x>} menunjukkan posisi entitas pada sumbu $X$ dari ruang gerak \textit{euclidean} $\mathbb{R}^3$. Hal yang sama juga berlaku untuk nilai \texttt{<posisi-y>} yang menyimpan posisi entitas pada sumbu $Y$ dan nilai \texttt{<posisi-z>} yang menyimpan posisi entitas pada sumbu $Z$. Nilai dari \texttt{<kecepatan-x>} menunjukkan kecepatan entitas pada sumbu $X$ dari ruang gerak \textit{euclidean} $\mathbb{R}^3$. Hal yang sama juga berlaku untuk nilai \texttt{<kecepatan-y>} yang menyimpan kecepatan entitas pada sumbu $Y$ dan nilai \texttt{<kecepatan-z>} yang menyimpan kecepatan entitas pada sumbu $Z$.

Berdasarkan batasan masalah yang sudah dibahas pada subbab \ref{sec:batasan}, nilai dari \texttt{<posisi-z>} akan dihiraukan sehingga entitas akan dianggap bergerak pada ruang $\mathbb{R}^2$. Selain itu, nilai dari \texttt{<kecepatan-x>}, \texttt{<kecepatan-y>}, dan \texttt{<kecepatan-z>} juga akan dihiraukan karena atribut-atribut tersebut tidak digunakan dalam proses identifikasi rombongan.

\section{Modul I/O}
\label{sec:impl-io}

Modul I/O akan diimplementasikan dalam berkas \texttt{io.cpp}. Berikut merupakan detil implementasi dari modul I/O berdasarkan 2 fungsionalitas modul I/O:

\begin{enumerate}
    \item \textbf{Input / Penerjemahan Masukan}
    
    Implementasi dari fungsi pembacaan masukan akan mengembalikan struktur data komposit bernama \texttt{Parameters}. Untuk memenuhi kebutuhan pembacaan masukan pengguna, modul I/O akan dibantu oleh sebuah pustaka C++ bernama \texttt{argparse}. Pustaka \texttt{argparse} merupakan sebuah pustaka yang memiliki fitur penerjemahan argumen yang diberikan melalui \textit{command line interface}.
    
    Pemilihan pustaka tersebut disebabkan oleh kemudahan penggunaan fitur penerjemahan dan konversi masukan menjadi tipe data tertentu serta memiliki fleksibilitas yang lebih baik dibandingkan pustaka lainnya dengan fitur serupa. 
    
    \item \textbf{Output / Pembuatan Berkas Hasil}
    
    Hasil identifikasi rombongan yang dihasilkan oleh modul rombongan akan ditulis dalam sebuah berkas teks yang memiliki format \texttt{.txt} dengan \textit{encoding} UTF-8. Terdapat 3 buah informasi yang akan ditulis pada berkas hasil identifikasi rombongan:
    
    \begin{enumerate}
        \item Nilai masukan yang diberikan pengguna melalui \textit{command line interface}.
        
        \item Nilai \textit{precision}, \textit{recall}, dan \textit{F1 score} dari hasil identifikasi rombongan.
        
        \item Kumpulan rombongan yang berhasil teridentifikasi.
    \end{enumerate}
    
    Sebuah rombongan yang berhasil teridentifikasi akan dicetak sebagai satu baris yang memiliki bentuk \texttt{<anggota>, <durasi>}. \texttt{<anggota>} menunjukkan kumpulan nomor identitas dari anggota rombongan yang dipisahkan menggunakan spasi. \texttt{<durasi>} menunjukkan kumpulan interval \textit{frame} konsekutif di mana rombongan tersebut terbentuk. Sebuah interval \textit{frame} ditulis sebagai sepasang \textit{frame} yang menunjukkan \textit{frame} mulai dan \textit{frame} berakhirnya rombongan. Fungsionalitas untuk mencetak berkas teks hasil identifikasi diimplementasikan menggunakan bantuan pustaka STL bawaan dari C++.
    
    Fungsi ini akan mencetak berkas yang memiliki nama yang sama dengan nama berkas sumber data pergerakan ditambah dengan \textit{epoch time} yang dihitung dari waktu eksekusi perangkat lunak. Berkas yang dihasilkan akan memiliki format \texttt{.txt} dengan \textit{encoding} UTF-8. Penamaan berkas dengan cara tersebut bertujuan untuk mempermudah penyimpanan banyak berkas hasil dengan sumber data yang sama. 
\end{enumerate}

\section{Modul Penerjemah}
\label{sec:impl-parser}

Modul penerjemah akan diimplementasikan dalam berkas \texttt{parser.cpp}. Seluruh struktur data yang diperlukan untuk mengimplementasikan modul ini tersedia melalui pustaka STL bawaan C++. Modul ini memiliki 3 fungsi utama berikut:

\begin{enumerate}
    \item \texttt{parse\_data}
    
    Fungsi ini bertugas untuk menerjemahkan data pergerakan pejalan kaki dari berkas teks menjadi struktur data komposit bernama \texttt{MovementData} yang membungkus kumpulan \texttt{Entity} dan kumpulan nilai \textit{frame} yang muncul pada data pergerakan. Pembuatan struktur data tersebut didasarkan oleh kebutuhan akan pencatatan durasi rombongan yang dibutuhkan oleh modul I/O.
    
    Seperti yang sudah dibahas pada rancangan perangkat lunak, perlu ditetapkan sebuah \textit{anchor value} yang berfungsi untuk menandakan kemunculan entitas pada sebuah \textit{frame}. Pada skripsi ini, \textit{anchor value} yang dipilih adalah nilai \texttt{NaN}. Nilai \texttt{NaN} dipilih untuk menghindari \textit{false negative} yang dapat disebabkan oleh nilai posisi yang variatif.

    Nilai \texttt{NaN} sebagai posisi dari sebuah entitas pada titik waktu $x$ menyatakan bahwa entitas tersebut tidak terekam pada titik waktu $x$, sehingga pemeriksaan kedekatan spasial dari entitas tersebut dengan entitas lain dapat dilewatkan. Hal tersebut akan meningkatkan efisiensi algoritma identifikasi rombongan dengan tidak melakukan perhitungan yang tidak perlu dilakukan. 
    
    \item \texttt{parse\_arguments}
    
    Fungsi ini bertugas untuk menerjemahkan masukan pengguna selain berkas data pergerakan menjadi parameter-parameter identifikasi rombongan yang akan digunakan oleh modul rombongan selama proses identifikasi rombongan berlangsung. Fungsi ini mengembalikan sebuah struktur data komposit bernama \texttt{Parameters}.
    
    \item \texttt{parse\_expected\_result}
    
    Fungsi ini bertugas untuk menerjemahkan data \textit{ground truth} agar hasil identifikasi rombongan bisa dievaluasi oleh modul evaluasi. Data \textit{ground truth} diambil dari sebuah berkas teks dengan format \texttt{.txt} dengan \textit{encoding} UTF-8 yang memiliki nama yang sama dengan nama berkas sumber data pergerakan. 
\end{enumerate}

\iffalse

\cristopher{ga penting?}

Untuk memenuhi kebutuhan tersebut, digunakan struktur data bernama \texttt{Map}. \texttt{Map} dipilih karena struktur data tersebut memiliki kemampuan untuk menyimpan dan mengembalikan pemetaan dengan kompleksitas waktu yang cenderung kecil. Pada implementasi dari modul ini, digunakan 2 buah \texttt{Map} di mana \texttt{Map} pertama digunakan untuk memetakan \textit{frame} pada entitas-entitas yang tercatat pada data pergerakan dan \texttt{Map} kedua digunakan untuk membalik pemetaan dari \texttt{Map} pertama menjadi pemetaan entitas pada \textit{frame} sehingga mempermudah proses pembuatan struktur data komposit yang akan dikembalikan.

\fi

\section{Modul Rombongan}
\label{sec:impl-rombongan}

Modul rombongan akan diimplementasikan dalam berkas \texttt{rombongan.cpp}. Seluruh struktur data yang diperlukan untuk mengimplementasikan modul ini tersedia melalui pustaka STL bawaan C++. Modul ini memiliki satu fungsi utama bernama \texttt{identify\_rombongan} yang merupakan implementasi konkrit dari algoritma identifikasi rombongan yang ditunjukkan oleh algoritma \ref{bab3:algoritma-identifikasi}. Fungsi tersebut akan mengembalikan kumpulan \texttt{Rombongan} yang dapat diidentifikasi dalam data pergerakan pejalan kaki menggunakan parameter yang diberikan oleh pengguna.

\section{Modul Spasial}

Modul spasial akan diimplementasikan dalam berkas \texttt{spatial.cpp}. Seluruh struktur data yang diperlukan untuk mengimplementasikan modul ini tersedia melalui pustaka STL bawaan C++. Terdapat dua fungsi utama yang diimplementasikan modul ini, yaitu fungsi untuk menghitung jarak \textit{dynamic time warping} dan menghitung \textit{cosine similarity}.

Fungsi untuk menghitung jarak \textit{dynamic time warping} akan diimplementasikan dalam fungsi bernama \texttt{calculate\_dtw\_distance} yang merupakan implementasi konkrit dari algoritma \textit{dynamic time warping} dari algoritma \ref{bab2:dtw-pseudocode}. Sedangkan fungsi untuk menghitung nilai \textit{cosine similarity} akan diimplementasikan dalam fungsi bernama \texttt{calculate\_cosine\_similarity} yang merupakan implementasi konkrit dari rumus \ref{bab3:cosine-similarity}.

\section{Struktur Data}
\label{sec:impl-struct}

Berdasarkan rancangan perangkat lunak yang telah dibuat serta pertimbangan implementasi yang telah dibahas sebelumnya, terdapat lima buah struktur data komposit yang perlu diimplementasikan. Seluruh struktur data akan diimplementasikan menggunakan kata kunci \texttt{struct}. Berikut merupakan detil implementasi dari ketiga struktur data tersebut:

\begin{enumerate}
    \item \textbf{Parameters}
    
    Struktur data ini menyimpan informasi mengenai masukan pengguna yang diberikan melalui \textit{command line interface}. rombongan yang diberikan oleh pengguna. Struktur data ini diimplementasikan sebagai properti-properti yang merupakan kumpulan-kumpulan tipe data primitif hasil kembalian dari modul I/O.
    
    Nilai dari $m$ dan $k$ diimplementasikan sebagai bilangan bulat menggunakan tipe data \texttt{int}. Nilai dari $r$ dan $\vartheta$ diimplementasikan sebagai bilangan desimal menggunakan tipe data \texttt{double}. Nama berkas data pergerakan yang akan digunakan diimplementasikan menggunakan tipe data \texttt{string} yang bersumber dari pustaka bawaan C++.
    
    \item \textbf{Arguments}
    
    \item \textbf{Entity}
    
    Struktur data ini mewakili sebuah entitas yang tercatat dalam data pergerakan yang diberikan. Struktur data ini memiliki 2 properti yaitu \texttt{id} yang menunjukkan nomor identitas unik yang dimiliki oleh entitas yang dimaksud dan \texttt{trajectories} yang merupakan catatan posisi dari entitas pada setiap \textit{frame} yang terdapat pada data pergerakan. Untuk mempermudah proses identifikasi rombongan, \texttt{trajectories} akan diimplementasikan menggunakan struktur data \texttt{Map} yang memetakan \textit{frame} pada posisi entitas.
    
    \item \textbf{Rombongan}
    
    Struktur data ini merupakan hasil dari identifikasi rombongan yang dikembalikan oleh modul Rombongan. Struktur data ini memiliki 3 buah properti yaitu \texttt{members} yang menyimpan nomor-nomor identitas entitas yang tergabung dalam rombongan, \texttt{start} yang menunjukkan \textit{frame} pertama di mana kumpulan entitas memenuhi syarat-syarat rombongan, dan \texttt{end} yang menunjukkan \textit{frame} terakhir di mana kumpulan entitas masih memenuhi syarat-syarat rombongan. Nilai \texttt{members} akan diimplementasikan menggunakan struktur data \texttt{vector}. Nilai dari \texttt{start} dan \texttt{end} diimplementasikan menggunakan tipe data \texttt{double}.
\end{enumerate}

\section{Pembangungan Perangkat Lunak}
\label{sec:build-tools}

Untuk membantu proses kompilasi kode perangkat lunak dari bahasa pemrograman C++ menjadi bahasa mesin, dibutuhkan sebuah perkakas bernama Make.

\section{Pengujian Unit}
\label{sec:unit-test}

Sebelum melalui pengujian hasil algoritma identifikasi rombongan, modul-modul dalam perangkat lunak akan melalui proses \textit{unit testing}. \textit{Unit testing} merupakan sebuah metode pengujian perangkat lunak di mana sebuah modul dalam perangkat lunak akan diuji secara terisolasi dari modul-modul lainnya. Pada \textit{unit testing}, diharapkan bahwa setiap modul perangkat lunak dapat menghasilkan perilaku tertentu yang diharapkan. Hal tersebut dapat dicapai dengan memberikan seperangkat masukan yang sudah disediakan sebelumnya.

Untuk melakukan \textit{unit testing} pada perangkat lunak identifikasi rombongan, digunakan sebuah perkakas pengujian bernama CTest yang terintergrasi langsung bersama CMake. Dengan menggunakan CTest, berkas-berkas \textit{unit testing} dapat dikelola dan dieksekusi dengan cara yang lebih mudah dibandingkan dengan melakukan pengujian secara manual. Seluruh berkas modul dalam perangkat lunak akan melalui tahap \textit{unit testing}, kecuali untuk berkas \texttt{rombongan.cpp} dan \texttt{rombongan.h}. Hal tersebut dilakukan mengingat tidak ada standar kebenaran absolut yang dapat digunakan untuk menguji berkas-berkas tersebut. Kedua berkas tersebut akan diuji secara langsung melalui bab Eksperimen.