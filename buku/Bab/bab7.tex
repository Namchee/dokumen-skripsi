\chapter{Simpulan}
\label{chap:simpulan}

\section{Pencapaian Tujuan Penelitian}
\label{sec:pencapaian-tujuan}

Seperti yang sudah dijabarkan pada subbab \ref{sec:tujuan}, tujuan dari penelitian ini adalah:

\begin{enumerate}
    \item \textbf{Membuat definisi pergerakan kolektif baru yang mampu mengatasi masalah identifikasi pada kasus perbedaan arah dan kecepatan}
    
    Subbab \ref{subsec:beda-arah} telah menjelaskan tentang masalah perbedaan arah yang dapat terjadi selama proses identifikasi pergerakan kolektif. Melalui analisis terhadap ukuran-ukuran yang terdapat pada data lintasan, diusulkan pemanfaatan ukuran arah lintasan untuk menyelesaikan masalah perbedaan arah pada proses identifikasi rombongan.
    
    Subbab \ref{subsec:beda-kecepatan} telah menjelaskan mengenai masalah perbedaan kecepatan yang dapat terjadi selama proses identifikasi pergerakan kolektif. Melalui analisis terhadap cara pengukuran kemiripan lintasan pada Subbab \ref{sec:kemiripan}, diusulkan penggunaan algoritma \textit{dynamic time warping} untuk menyelesaikan masalah perbedaan kecepatan pada proses identifikasi rombongan.
    
    Berdasarkan dua usulan penyelesaian tersebut, dibuatlah sebuah definisi pergerakan kolektif baru bernama rombongan yang dinyatakan pada Subbab \ref{sec:definisi}.
    
    \item \textbf{Membuat dan mengimplementasikan algoritma yang dapat mengidentifikasi kelompok pergerakan kolektif yang sesuai dengan definisi formal yang telah dibuat}
    
    Algoritma identifikasi pergerakan kolektif yang sesuai dengan definisi formal rombongan dapat dilihat melalui Algoritma \ref{bab3:algoritma-identifikasi} yang dijabarkan melalui subbab \ref{sec:algoritma}. Cara kerja algoritma identifikasi rombongan dapat dilihat melalui Subbab \ref{sec:algorithm-example}. Karena cara kerja algoritma identifikasi rombongan yang dapat menghasilkan banyak rombongan yang redundan, dibuatlah sebuah algoritma pengurangan redundansi yang dapat dilihat melalui Algoritma \ref{bab3:redundansi}.
    
    Rancangan perangkat lunak identifikasi rombongan dibahas pada Bab \ref{chap:perancangan}. Bab tersebut memuat pertimbangan-pertimbangan serta rencana alur kerja perangkat lunak identifikasi rombongan. Detil implementasi perangkat lunak dibahas melalui Bab \ref{chap:implementasi}.
    
    \item \textbf{Melakukan pengujian terhadap definisi formal pergerakan kolektif yang sudah dibuat secarakuantitatif dan kualitatif}
    
    Pengujian terhadap algoritma identifikasi rombongan diukur menggunakan pengujian kualitatif dan kuantitatif. Pengujian kualitatif dilakukan melalui analisis terhadap visualisasi hasil identifikasi pejalan kaki. Pengujian kuantitatif dilakukan melalui perbandingan hasil identifikasi manusia dan hasil identifikasi yang dihasilkan perangkat lunak dengan menghitung nilai \textit{precision}, \textit{recall}, dan \textit{F1 score}.
    
    Hasil pengujian kuantiatif menunjukkan bahwa upaya identifikasi pergerakan kolektif yang sesuai dengan definisi rombongan memiliki nilai \textit{precision} yang cukup rendah dan nilai \textit{recall} yang tinggi pada data pergerakan bidireksional. Pengujian definisi pada data pergerakan multidireksional menghasilkan nilai \textit{precision} yang cukup tinggi namun disertai dengan nilai \textit{recall} yang rendah. Memperketat syarat perbedaan arah maksimum meningkatkan nilai \textit{F1 score} secara konsisten pada setiap data pergerakan yang diuji.
    
    Hasil pengujian kualitatif menunjukkan masalah-masalah yang dialami oleh definisi rombongan. Masalah utama yang dialami pada proses identifikasi rombongan yang sesuai dengan definisi rombongan adalah perhitungan perbedaan arah yang terlalu ketat dan kaku sehingga menyebabkan rombongan tidak teridentifikasi apabila terdapat perubahan arah sekecil apapun. Terlepas dari masalah tersebut, pengujian kualitatif menunjukkan bahwa definisi rombongan dapat mengidentifikasi rombongan yang memiliki kasus perbedaan kecepatan dengan baik.
\end{enumerate}

Berdasarkan jawaban-jawaban di atas, dapat disimpulkan bahwa tujuan penelitian yang telah dijabarkan sebelumnya dapat dicapai seluruhnya. Dengan demikian, dapat dibuat sebuah penjabaran akan jawaban dari masalah-masalah yang hendak dipecahkan oleh penelitian ini.

\section{Jawaban Atas Masalah Penelitian}
\label{sec:jawaban-masalah}

Seperti yang sudah disebutkan melalui Subbab \ref{sec:rumusan}, masalah utama yang hendak dipecahkan oleh skripsi ini adalah \textit{seperti apakah definisi pergerakan kolektif yang dapat mengatasi masalah identifikasi pada data pergerakan yang memiliki kasus perbedaan arah dan kecepatan}. Pemecahan masalah tersebut dilakukan dengan melakukan studi pustaka mengenai ukuran-ukuran yang terdapat dalam sebuah data lintasan. Studi pustaka mengenai lintasan dibahas secara detil melalui Bab \ref{chap:teori}.

Detil dari masalah perbedaan arah dan kecepatan dibahas pada Bab \ref{chap:analisis}. Melalui pembahasan yang dilakukan di bab tersebut, diketahui bahwa arah lintasan dapat digunakan untuk memecahkan masalah perbedaan arah entitas. Untuk mengatasi masalah perbedaan kecepatan entitas, digunakan algoritma \textit{dynamic time warping} sebagai pengganti dari penggunaan \textit{euclidean distance} untuk mengukur jarak antara dua buah entitas. Dari usulan solusi untuk memecahkan masalah beda arah dan kecepatan, lahirlah sebuah definisi pergerakan kolektif baru bernama rombongan. Setelah definisi rombongan rampung, dibuatlah sebuah algoritma yang mampu mengidentifikasi rombongan sesuai dengan definisi tersebut. Algoritma identifikasi rombongan dapat dilihat melalui Subbab \ref{sec:algoritma} beserta cara kerja algoritma yang dapat dilihat melalui Subbab \ref{sec:algorithm-example}. Algoritma identifikasi rombongan kemudian dirancang dan diimplementasikan menjadi sebuah perangkat lunak.

Untuk menguji definisi pergerakan kolektif baru yang dibuat, dilakukan uji relevansi hasil identifikasi secara kualitatif dan kuantitatif. Pengujian kuantitatif dilakukan dengan menghitung relevansi hasil identifikasi perangkat lunak dengan hasil identifikasi yang dilakukan oleh manusia melalui nilai \textit{precision}, \textit{recall}, dan \textit{F1 score}. Pengujian kualitatif dilakukan melalui pengamatan terhadap hasil visualisasi rombongan pada rekaman video pejalan kaki. Untuk membantu kebutuhan akan visualisasi hasil identifikasi pergerakan kolektif, digunakan sebuah perangkat lunak lain yang merupakan perangkat lunak diluar lingkup pengerjaan penelitian ini. Perangkat lunak visualisasi akan menghasilkan video rekaman yang telah dimodifikasi sehingga setiap rombongan yang teridentifikasi ditandai menggunakan model warna.

Hasil pengujian kuantitatif menunjukkan peningkatan nilai relevansi hasil identifikasi perangkat lunak apabila syarat perbedaan arah diperketat secara konsisten untuk setiap data pergerakan yang diuji. Pengujian kuantitatif juga menunjukkan pengaruh parameter lain pada relevansi hasil identifikasi perangkat lunak seperti parameter jarak \textit{dynamic time warping} maksimum dan durasi interval waktu minimum. Hasil pengujian kualitatif menunjukkan masalah ketatnya syarat perbedaan arah yang menyebabkan banyaknya rombongan yang gagal teridentifikasi dan menyebabkan banyak identifikasi subrombongan yang tidak relevan. Terlepas dari masalah tersebut, hasil pengujian kualitatif menunjukkan bahwa definisi rombongan dapat mengatasi masalah perbedaan kecepatan. Bukti dari penyelesaian tersebut dapat dilihat pada visualisasi hasil identifikasi perangkat lunak yang menunjukkan bahwa rombongan yang memiliki anggota dengan kecepatan yang berbeda dapat diidentifikasi dengan baik dan sedini mungkin.

\section{Saran Penelitian Lanjutan}
\label{sec:saran}

Berdasarkan proses dan hasil penelitian, terdapat beberapa poin-poin menarik yang dapat dijadikan sebagai bahan penelitian lanjutan mengenai pola pergerakan kolektif rombongan. Berikut merupakan poin-poin yang dianggap dapat menjadi bahan penelitian lanjutan:

\begin{enumerate}
    \item Melakukan pengujian lanjutan pada lebih banyak sampel data pergerakan, khususnya pada sampel-sampel yang memiliki tingkat kepadatan entitas yang variatif, memiliki pola pergerakan yang tidak umum seperti adanya perubahan kecepatan dan arah yang ekstrim, serta pada data pergerakan di mana entitas yang diamati bukan manusia.
    \item Meningkatkan kualitas hasil identifikasi pergerakan kolektif yang dilakukan oleh manusia.
    \item Melakukan pengujian mengenai pengaruh parameter identifikasi lainnya seperti jumlah entitas minimum $m$ dan jumlah kedekatan minimum $n$.
    \item Melakukan optimisasi terhadap kompleksitas algoritma identifikasi rombongan.
    \item Meningkatkan fleksibilitas perhitungan perbedaan arah agar dapat mengatasi masalah perbedaan arah yang kecil dengan frekuensi yang tinggi.
    \item Mengurangi redundansi yang terdapat algoritma identifikasi rombongan sehingga perangkat lunak tidak membutuhkan algoritma tambahan untuk mengurangi redundansi hasil identifikasi.
    \item Mengimplementasikan pemrosesan paralel pada algoritma identifikasi rombongan.
\end{enumerate}