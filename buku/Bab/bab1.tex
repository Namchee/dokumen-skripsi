%versi 2 (8-10-2016) 
\chapter{Pendahuluan}
\label{chap:intro}
   
\section{Latar Belakang}
\label{sec:label}

Bergerak merupakan sebuah aktivitas yang selalu kita temui setiap saat. Berdasarkan Kamus Besar Bahasa Indonesia, bergerak merupakan sebuah kegiatan berpindah dari tempat atau kedudukan semula menuju tempat atau kedudukan lain\footnote{Badan Pengembangan dan Pembinaan Bahasa Kementrian Pendidikan dan Kebudayan, \textit{Bergerak}, diakses pada tanggal 23 Desember 2020, \url{https://kbbi.kemdikbud.go.id/entri/bergerak}.}. Segala sesuatu yang ada di sekitar kita selalu bergerak untuk mencapai tujuan tertentu. Manusia selalu bergerak untuk melakukan aktivitas sehari-hari. Berbagai jenis hewan melakukan migrasi untuk mencari lingkungan baru yang lebih mampu untuk menunjang kehidupan. Tanaman dapat bergerak melalui fototropisme yang menyebabkan tanaman bertumbuh mengikuti arah sinar matahari. Bahkan, bumi selalu berputar mengelilingi porosnya yang menyebabkan pergantian hari. Dapat dikatakan bahwa bergerak merupakan aktivitas yang tak dapat dilepaskan dari kehidupan sehari-hari. Hal tersebut menumbuhkan rasa ketertarikan manusia untuk mengumpulkan, menyelidiki, serta mempelajari segala aspek mengenai pergerakan. Sayangnya, sedikitnya sumber dan cara mendapatkan data menghambat penelitian mengenai pergerakan di masa lalu.

\begin{figure}[h]
    \centering
    \begin{subfigure}[b]{0.45\textwidth}
        \includegraphics[width=\textwidth, height=4.5cm]{Gambar/bab1:manusia.jpg}
        \caption{Manusia bergerak untuk memenuhi kebutuhan sehari-hari\protect\footnotemark}
        \label{bab1:manusia}
    \end{subfigure}
    \begin{subfigure}[b]{0.45\textwidth}
        \includegraphics[width=\textwidth, height=4.5cm]{Gambar/bab1:sunflower.jpg}
        \caption{Bunga matahari bertumbuh mengikuti arah sinar matahari\protect\footnotemark}
        \label{bab1:sunflower}
    \end{subfigure}
    \caption[Aktivitas pergerakan]{
    Aktivitas pergerakan yang dilakukan oleh berbagai entitas di sekitar kita}
    \label{bab1:pergerakan}
\end{figure}

\footnotetext{Alvin Mamudov,  2017, diakses pada tanggal 4 Januari 2021, \url{https://unsplash.com/photos/FlLHbmF3AHc}.}

\footnotetext{Lisa Pellegrini, 2016, diakses pada tanggal 4 Januari 2021, \url{https://unsplash.com/photos/XCvy_eufErI}.}

Seiring berjalannya waktu, kemunculan teknologi modern seperti \textit{Geographic Information System} (GIS) menyebabkan data pergerakan yang bersumber dari berbagai entitas menjadi semakin banyak dan semakin mudah didapatkan. Hal tersebut memicu kembalinya pertumbuhan minat penelitian mengenai pergerakan oleh berbagai peneliti dalam berbagai bidang. Terdapat banyak hal menarik yang dapat dimanfaatkan melalui analisis terhadap data pergerakan. Salah satu contoh nyata dari pemanfaatan analisis data pergerakan adalah analisis pergerakan bebek domestik, di mana melalui data pergerakan bebek, virolog dapat mengidentifikasi daerah-daerah yang memiliki risiko persebaran flu burung yang tinggi \cite{diann:01:movement-analysis}.

Dalam sebuah pergerakan, setiap entitas yang bergerak memiliki lintasan. Lintasan merupakan jalur yang dilalui oleh entitas selama melakukan pergerakan dalam rentang waktu tertentu. Data mengenai lintasan yang ditempuh oleh entitas yang bergerak bersifat kontinu. Data lintasan dapat diperoleh melalui berbagai cara, di mana hal tersebut ditentukan oleh tipe entitas, lingkungan tempat terjadinya pergerakan, teknologi yang digunakan, dan lain sebagainya \cite{wiratma:trajectory}. Sekarang ini, data mengenai lintasan biasanya diperoleh melalui sistem modern seperti sistem Argos-Doppler dan \textit{global positioning system} (GPS) yang lazim ditemukan pada telepon pintar, di mana keduanya sama-sama memanfaatkan teknologi satelit \cite{carter:argos}. Kedua sistem tersebut memiliki beberapa kelebihan dibandingkan penggunaan cara-cara tradisional seperti memiliki akurasi yang jauh lebih baik, mampu memperoleh lebih banyak data lintasan dari berbagai entitas sekaligus, dapat dikustomisasi sesuai kebutuhan, dan masih banyak lagi.

\begin{figure}[h]
    \centering
    \includegraphics[width=0.8\textwidth]{Gambar/bab1:sumber-data.png}
    \caption[Teknologi pengambilan data lintasan]{Pengambilan data lintasan melalui pemanfaatan teknologi modern. Kiri ke kanan: Sistem Argos-Doppler, \textit{Global positioning system} (GPS)}
    \label{bab1:sumber-data}
\end{figure}

Sayangnya, perkembangan teknologi tetap tidak mampu untuk merekam data lintasan secara sempurna. Hal tersebut menyebabkan data lintasan yang diperoleh tidak bersifat kontinu seperti yang diharapkan, melainkan bersifat diskrit. Berdasarkan sifat tersebut, data lintasan yang diperoleh akan direpresentasikan sebagai catatan posisi dari entitas yang bergerak yang diurutkan berdasarkan titik waktu. Secara formal, lintasan merupakan himpunan dari pasangan posisi-waktu $(p_0, t_0), (p_1, t_1), \ldots, (p_x, t_x)$, di mana $p_i$ merupakan posisi entitas yang relatif terhadap ruang gerak entitas pada titik waktu $t_i$. Pada umumnya, entitas akan bergerak dalam sebuah ruang \textit{euclidean} dua dimensi atau tiga dimensi yang masing-masing dapat direpresentasikan sebagai 
$\mathbb{R}^2$ dan $\mathbb{R}^3$. \iffalse \lionov{disebut di sini bahwa ruang gerak biasanya R2 atau R3}. \fi

Setiap lintasan memiliki atribut-atribut dengan nilai tertentu sebagai karakteristik yang membedakan satu lintasan dengan lintasan lain. Atribut-atribut tersebut kemudian dapat diolah menjadi ukuran lintasan yang memiliki kegunaan yang lebih spesifik. Analisis terhadap data pergerakan selalu memanfaatkan setidaknya salah satu ukuran yang terdapat pada data lintasan. Sebagai contoh, kecepatan dapat digunakan untuk mengukur pengaruh angin pada pergerakan burung \cite{safi:speed}.

\iffalse \lionov{jadiin paragraf baru, kan ini beda topik} \fi

Terdapat berbagai jenis analisis yang dapat dilakukan pada data pergerakan seperti segmentasi lintasan \cite{mann:01:segmentation}, pengukuran kemiripan lintasan \cite{rote:01:hausdorff, alt:01:frechet, muller:dtw}, \textit{clustering} pada entitas yang bergerak \cite{lee:01:clustering}, dan identifikasi pola pergerakan kolektif yang menjadi fokus utama pada skripsi ini. Tujuan dari analisis pola pergerakan kolektif adalah mengidentifikasi kelompok pergerakan yang terbentuk dari entitas-entitas yang bergerak bersama dalam rentang waktu yang cukup lama. Analisis pola pergerakan kolektif memiliki pemanfaatan dalam berbagai bidang. Sebagai contoh, identifikasi pola pergerakan kolektif dapat dimanfaatkan pada bidang keamanan untuk mengidentifikasi pergerakan mencurigakan dari sekelompok orang \cite{makris:01:security}.

\iffalse 

\lionov{sebaiknya ada ilustrasi kayak Figure 1.7 di thesis, buat memperjelas apa itu pergerakan kolektif} \cristopher{sedang digambar}.

\fi

\begin{figure}[h]
    \centering
    \begin{subfigure}[h]{0.45\textwidth}
        \centering
        \includegraphics[width=\textwidth, height=4.5cm]{Gambar/bab1:army.jpg}
        \caption{Kumpulan tentara bergerak bersama membentuk sebuah regu\protect\footnotemark}
    \end{subfigure}
    \begin{subfigure}[h]{0.45\textwidth}
        \centering
        \includegraphics[width=\textwidth, height=4.5cm]{Gambar/bab1:wildebeest.jpg}
        \caption{Kawanan \textit{wildebeest} bergerak bersama untuk melakukan migrasi musiman\protect\footnotemark}
        \label{bab1:wildebeest}
    \end{subfigure}
    \caption[Pergerakan kolektif dunia nyata]{Pergerakan kolektif yang sering kita temui di dunia nyata}
    \label{bab1:collective-movement}
\end{figure}

\footnotetext{Damon On Road, \textit{C\'{e}r\'{e}monie militaire du Man\`{e}ge militaire du Qu\'{e}bec}, 2020, diakses pada tanggal 4 Januari 2021, \url{https://unsplash.com/photos/nmePDwaW9I8}}

\footnotetext{Jorge Tung, \textit{Great wildebeest migration crossing Mara river at Serengeti National Park --- Tanzania}, 2019, diakses pada tanggal 23 Desember 2020, \url{https://unsplash.com/photos/1pZJqQlgpsY}}

Ada berbagai macam definisi formal yang sudah dibuat untuk mengidentifikasi pola pergerakan kolektif seperti \textit{flock} \cite{cao:flock, gudmundsson:flock}, \textit{convoy} \cite{jeung:convoys}, \textit{group} \cite{buchin:group, yida:group}, dan masih banyak lagi. Seluruh definisi formal tersebut bergantung pada \textit{size}\iffalse \lionov{ini jadi aneh, pake {\it size} aja deh} kelompok \fi, kedekatan spasial, dan durasi temporal untuk melakukan identifikasi pergerakan kolektif pada sekelompok entitas yang bergerak bersama. \textit{Size} menentukan jumlah anggota minimum yang harus tergabung dalam sebuah pergerakan kolektif. Kedekatan spasial menentukan batas maksimum jarak antar anggota kelompok. Durasi temporal menentukan durasi minimum pergerakan bersama dari seluruh anggota pergerakan kolektif.

Namun, seluruh definisi formal yang ada tidak mampu melakukan identifikasi yang akurat pada dua kasus pergerakan yang lazim terjadi pada di dunia nyata. Masalah pertama, definisi-definisi formal pergerakan kolektif yang sudah ada dapat menghasilkan hasil yang tidak tepat pada kasus perbedaan arah. Sebagai contoh, terdapat sebuah pergerakan kolektif yang terdiri dari tiga buah anggota yang bergerak ke arah kanan dan sebuah entitas yang bergerak ke arah kiri di mana keduanya memiliki kecepatan yang lambat. Ketika entitas yang bergerak ke kiri berpapasan dengan pergerakan kolektif yang bergerak ke arah kanan, kecepatan yang lambat dapat membuat entitas yang bergerak ke arah kiri memiliki jarak yang dekat dengan setiap anggota pergerakan kolektif selama kedua pihak berpapasan. Kondisi tersebut dapat menyebabkan definisi-definisi pergerakan kolektif yang sudah ada melakukan kesalahan identifikasi dengan mengikutsertakan entitas yang berpapasan sebagai anggota pergerakan kolektif selama kedua pihak berpapasan. Hal tersebut disebabkan karena kondisi tersebut memenuhi syarat kedekatan spasial dan durasi temporal dari pergerakan kolektif, di mana entitas yang bergerak akan memiliki jarak yang dekat dengan anggota pergerakan kolektif dalam rentang waktu yang cukup lama mengingat keduanya memiliki kecepatan yang rendah. Hasil tersebut tentunya bertentangan dengan hasil identifikasi yang dilakukan oleh manusia mengingat secara nalar, entitas yang berpapasan tersebut memiliki perbedaan arah yang besar dengan anggota-anggota pergerakan kolektif sehingga entitas tersebut tidak bisa diidentifikasi sebagai anggota pergerakan kolektif.

Masalah kedua, definisi-definisi formal pergerakan kolektif yang sudah ada akan mengalami kesulitan dalam menyimpulkan keanggotaan pergerakan kolektif pada kasus-kasus pergerakan di mana entitas-entitas yang bergerak memiliki kecepatan yang variatif. Sebagai contoh, terdapat sebuah pergerakan kolektif yang terdiri dari tiga buah anggota di mana terdapat satu anggota dari pergerakan kolektif yang memimpin pergerakan kolektif dengan berjalan mendahului kedua anggota lainnya. Karena entitas tersebut terus memimpin berjalan dengan menatap ke depan, entitas tersebut secara tidak sadar memiliki kecepatan yang lebih cepat dibandingkan dua entitas yang lain dan perlahan-lahan meninggalkan entitas-entitas lain. Kondisi tersebut menyebabkan entitas yang memimpin memiliki perbedaan jarak yang cukup jauh dengan kedua anggota lainnya. Setelah beberapa saat, entitas yang memimpin berhenti kemudian menoleh ke belakang untuk memastikan apakah dirinya dan kedua entitas lainnya masih bergerak secara kolektif. Karena perbedaan jarak yang jauh, entitas yang memimpin memutuskan untuk memperlambat gerakan atau menunggu entitas-entitas lainnya untuk menyusul sampai jarak tertentu kemudian menyesuaikan kecepatannya dengan kedua entitas lain untuk kembali bergerak secara kolektif dengan anggota lain. Pada kasus tersebut, definisi-definisi formal pergerakan kolektif yang sudah ada dapat menghasilkan identifikasi di mana entitas yang memimpin tidak tergabung sebagai anggota pergerakan kolektif selama beberapa saat. Hal tersebut disebabkan karena entitas yang memimpin memiliki jarak yang jauh dengan anggota-anggota pergerakan kolektif lainnya selama beberapa saat sehingga syarat kedekatan spasial dan durasi temporal tidak terpenuhi. Hasil tersebut bertentangan dengan hasil identifikasi pergerakan kolektif yang dilakukan oleh manusia, di mana entitas yang memimpin akan selalu tergabung sebagai anggota pergerakan kolektif mengingat kasus serupa lazim terjadi dalam sebuah rombongan pejalan kaki dunia nyata. Dua masalah tersebut mendorong perlunya perluasan ukuran lintasan yang digunakan dalam proses identifikasi terhadap sebuah pola pergerakan kolektif serta pembuatan definisi formal pergerakan kolektif baru yang mampu mengatasi masalah identifikasi pergerakan kolektif pada kasus-kasus serupa. 

Pada skripsi ini, akan dibuat sebuah definisi formal pergerakan kolektif baru yang akan memperluas ukuran-ukuran penentu yang digunakan untuk mengidentifikasi pola pergerakan kolektif. Setelah definisi formal selesai dibuat, definisi tersebut akan diuji efektivitasnya melalui sebuah eksperimen dengan mengidentifikasi pola pergerakan kolektif yang sesuai dengan definisi tersebut pada data pejalan kaki di dunia nyata yang tersedia melalui sumber data publik di internet. Hasil identifikasi dari definisi tersebut kemudian akan dibandingkan dengan hasil identifikasi pergerakan kolektif yang dilakukan oleh manusia yang diikutsertakan pada data lintasan. Untuk memenuhi kebutuhan tersebut, sebuah algoritma akan dikembangkan untuk mengidentifikasi pola pergerakan kolektif yang sesuai dengan definisi formal yang dibuat. Algoritma tersebut kemudian akan diimplementasikan menjadi sebuah perangkat lunak menggunakan bahasa pemrograman C++. Setelah proses identifikasi selesai, hasil identifikasi rombongan akan divisualiasikan dengan menggunakan sebuah model pewarnaan di dalam video rekaman data pejalan kaki.

\iffalse

\lionov{tambahin bahwa setelah itu akan dilakukan eksperimen dengan data nyata lintasan yang tersedia di internet (lihat bab 5 di thesis) dan akan dibandingkan hasilnya dengan hasil dari eksperimen sebelumnya (jangna lupa jelasin bahwa di eksperimen sebelumnya, yang dibandingkan itu cuma grup, gak ngebandingin tempat dan waktu). Terus harus disinggung bahwa di data eksperimen itu, udah ada grup yang ditentukan oleh manusia, jadi nanti hasil program akan dibandingkan dengan kunci jawabannya.} \cristopher{Di atas kan udah saya sebuat kalo identifikasinya akan diuji dengan dibandingkan, kalo gitu kudu rephrase atau biarin ajah ko?}

\fi

\section{Rumusan Masalah}
\label{sec:rumusan}

Berdasarkan uraian pada bagian sebelumnya, berikut merupakan masalah-masalah yang hendak diselesaikan oleh skripsi ini:
\iffalse

\lionov{masalah pertama itu ukuran apa saja yang bisa digunakan, kedua bagaimana membuat model pergerakan kolektif yang memanfaatkan ukuran lintasan, yang ketiga bagaimana membuat algoritmanya. Soalnya yang poin 2 dan 3 kan gak dilakukan, elu gak ngebahas berbagai definisi formal, ukuran, teknik, dan algoritma (karena pertanyaannya ``apa aja''}

\fi
\begin{enumerate}
    \item Apa saja ukuran-ukuran yang terdapat dalam data lintasan yang bisa digunakan untuk membuat sebuah definisi pergerakan kolektif?
    \item Bagaimana cara membuat definisi pergerakan kolektif yang memanfaatkan ukuran-ukuran yang terdapat dalam data lintasan?
    \item Bagaimana cara membuat algoritma untuk mengidentifikasi pergerakan kolektif berdasarkan definisi pergerakan kolektif yang dibuat?
    \item Bagaimana cara mengimplementasikan algoritma untuk mengidentifikasi pergerakan kolektif berdasarkan definisi pergerakan kolektif yang dibuat?
    \item Bagaimana cara memvisualisasikan hasil identifikasi rombongan dalam data pejalan kaki?
    \item Seberapa efektifkah definisi formal pergerakan kolektif yang dibuat dalam mengidentifikasi pergerakan-pergerakan kolektif dalam data pejalan kaki di dunia nyata?
\end{enumerate}

\section{Tujuan}
\label{sec:tujuan}  

Berdasarkan masalah-masalah yang telah dirumuskan pada bagian sebelumnya, maka skripsi ini bertujuan untuk:

\begin{enumerate}
    \item Menentukan ukuran-ukuran lintasan yang dapat digunakan untuk membuat sebuah definisi pola pergerakan kolektif baru yang mampu mengatasi masalah identifikasi pada kasus perbedaan arah dan kecepatan.
    \item Membuat definisi pergerakan kolektif baru yang mampu mengatasi masalah identifikasi pada kasus perbedaan arah dan kecepatan.
    \item Membuat algoritma yang dapat mengidentifikasi kelompok pergerakan kolektif yang sesuai dengan definisi formal yang telah dibuat pada sebuah data pejalan kaki di dunia nyata.
    \item Melakukan implementasi terhadap algoritma identifikasi pergerakan kolektif berdasarkan definisi baru yang dibuat.
    \item Melakukan visualisasi hasil identifikasi rombongan menggunakan model pewarnaan dalam video rekaman data pejalan kaki.
    \item Mengukur efektivitas dan ketepatan dari definisi pergerakan kolektif yang dibuat dalam mengidentifikasi pergerakan-pergerakan kolektif dalam data pejalan kaki di dunia nyata.
\end{enumerate}

\section{Batasan Masalah}
\label{sec:batasan}

Dalam pengerjaannya, skripsi ini hanya akan membahas masalah-masalah berikut:

\begin{enumerate}
    \item Data pergerakan yang akan dianalisis bersumber dari pergerakan entitas dalam ruang \textit{euclidean} dua dimensi $\mathbb{R}^2$.
    \item Hasil identifikasi pola pergerakan kolektif yang dilakukan oleh manusia yang diikutsertakan pada data lintasan merupakan sumber kebenaran yang tidak dapat diubah.
    \item Perangkat lunak yang digunakan untuk melakukan visualisasi hasil identifikasi rombongan merupakan perangkat lunak yang sudah tersedia dan bukan merupakan bagian dari implementasi perangkat lunak dalam skripsi ini.
    \item Evaluasi hasil eksperimen tidak akan memperhitungkan kompleksitas algoritma yang digunakan, melainkan hanya akan mengukur akurasi identifikasi pola pergerakan kolektif dibandingkan dengan identifikasi pola pergerakan kolektif yang dilakukan oleh manusia.
\end{enumerate}

\section{Metodologi}
\label{sec:metlit}

Metodologi penelitian yang akan digunakan dalam skripsi ini adalah sebagai berikut:

\begin{enumerate}
    \item Melakukan studi pustaka mengenai berbagai ukuran dan atribut yang terdapat pada sebuah lintasan dari entitas yang bergerak.
    \item Melakukan studi pustaka mengenai berbagai definisi formal pola pergerakan kolektif yang sudah dibuat sebelumnya.
    \item Melakukan studi pustaka mengenai ukuran kemiripan lintasan serta algoritma yang dapat digunakan untuk mengukur kemiripan lintasan.
\end{enumerate}

\section{Sistematika Pembahasan}
\label{sec:sispem}

Skripsi ini akan ditulis dalam enam bab yang masing-masing akan berisi:

\begin{enumerate}
    \item Bab 1 Pendahuluan
    
    Bab 1 akan berisi pembahasan mengenai latar belakang tentang pergerakan, lintasan, pergerakan kolektif, serta masalah-masalah yang terdapat pada definisi-definisi pergerakan kolektif sebelumnya. Bab ini juga membahas rumusan masalah, tujuan yang hendak dicapai, batasan masalah, metodologi penelititan, serta sistematika penulisan dari skripsi ini.
    
    \item Bab 2 Dasar Teori
    
    Bab 2 akan berisi dasar teori mengenai lintasan, sumber data lintasan, ukuran yang terdapat pada lintasan, penghitungan kemiripan lintasan, algoritma \textit{dynamic time warping} yang dapat digunakan untuk mengukur kemiripan antara dua buah lintasan, serta pergerakan kolektif secara umum. Pada bab ini juga akan dibahas secara singkat mengenai definisi pergerakan kolektif yang sudah ada.
    
    \item Bab 3 Analisis
    
    Bab 3 akan berisi pembahasan mengenai masalah-masalah yang terdapat pada definisi-definisi formal pergerakan kolektif yang sudah ada serta cara untuk mengatasi masalah-masalah tersebut. Selain itu, bab ini juga akan membahas mengenai definisi formal pergerakan kolektif baru yang dibuat dan algoritma yang dapat digunakan untuk mengidentifikasi pergerakan kolektif berdasarkan definisi tersebut.
    
    \item Bab 4 Perancangan
    
    Bab 4 akan berisi pembahasan mengenai rancangan perangkat lunak yang merupakan implementasi algoritma identifikasi rombongan yang diusulkan. Secara spesifik, bab ini akan membahas mengenai modul-modul perangkat lunak yang akan dibuat serta struktur-struktur data yang akan digunakan selama proses identifikasi rombongan dilakukan.
    
    \item Bab 5 Pengujian
    
    Bab 5 akan berisi
    
    \item Bab 6 Simpulan
    
    Bab 6 akan berisi
\end{enumerate}
